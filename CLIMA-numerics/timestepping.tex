\documentclass{article}

\newcommand{\hlc}[2][yellow]{ {\sethlcolor{#1} \hl{#2}} }

\usepackage[T1]{fontenc}
\usepackage[utf8]{inputenc}
\usepackage{times}

\usepackage[font=small,labelfont=bf,tableposition=top]{caption}
\usepackage{graphicx}
\usepackage{natbib} 

\usepackage{amsmath}
\usepackage{amsfonts}
\usepackage{amssymb}
\usepackage{color, soul}
\usepackage{hyperref}
\usepackage{algorithmicx}
\usepackage{algpseudocode}
\usepackage{subfigure}
\usepackage{stmaryrd}

\renewcommand{\vec}[1]{\boldsymbol{{#1}}} 
\newcommand{\duesoon}[1]{{\sethlcolor{green}\hl{#1}}}
\usepackage{mathrsfs}


\newtheorem{theorem}{Theorem}
\newtheorem{acknowledgement}[theorem]{Acknowledgement}
\newtheorem{algorithm}[theorem]{Algorithm}
\newtheorem{axiom}[theorem]{Axiom}
\newtheorem{case}[theorem]{Case}
\newtheorem{claim}[theorem]{Claim}
\newtheorem{conclusion}[theorem]{Conclusion}
\newtheorem{condition}[theorem]{Condition}
\newtheorem{conjecture}[theorem]{Conjecture}
\newtheorem{corollary}[theorem]{Corollary}
\newtheorem{criterion}[theorem]{Criterion}
\newtheorem{definition}[theorem]{Definition}
\newtheorem{example}[theorem]{Example}
\newtheorem{exercise}[theorem]{Exercise}
\newtheorem{lemma}[theorem]{Lemma}
\newtheorem{notation}[theorem]{Notation}
\newtheorem{problem}[theorem]{Problem}
\newtheorem{proposition}[theorem]{Proposition}
\newtheorem{remark}[theorem]{Remark}
\newtheorem{solution}[theorem]{Solution}
\newtheorem{summary}[theorem]{Summary}
\newenvironment{proof}[1][Proof]{\textbf{#1.} }{\ \rule{0.5em}{0.5em}}

\newtheorem{guess}{Definition}
\newcommand{\comment}[1] {}
\newcommand{\Norder} {N}
\newcommand{\order}{\mathcal{O}}
\newcommand{\Npoints} {N_p}
\newcommand{\Nfaces} {N_{f}}
\newcommand{\Nelements} {N_e}

\newcommand{\eps}{\varepsilon}
\newcommand{\Dweak}{\wt{D}}
\newcommand{\diff}[2] {\frac{\partial #1}{\partial #2}}
\newcommand{\dxx}[2] {\frac{\partial^2 #1}{\partial {#2}^2}}
\newcommand{\difft}[2] {\frac{d #1}{d #2}}
\newcommand{\dxxt}[2] {\frac{d^2 #1}{d {#2}^2}}
\newcommand{\lagrange}[1] {\frac{d #1}{dt}}
\newcommand{\lebesgue}{\parallel I \parallel}
\newcommand{\polysp}{\mathcal{P}_N}
\newcommand{\laplacian}{\nabla^2}
\newcommand{\divergence}{\nabla \cdot}
\newcommand{\inte}{\int_{\mbox{\footnotesize ${\Omega_e}$}}}
\newcommand{\intb}{\int_{\mbox{\footnotesize ${\Gamma_e}$}}}
\newcommand{\intce}{\int_{\mbox{\footnotesize ${\widehat{\Omega}_e}$}}}
\newcommand{\intcb}{\int_{\mbox{\footnotesize ${\widehat{\Gamma}_e}$}}}
\newcommand{\intg}{\int_{\mbox{\footnotesize ${\Omega}$}}}
\newcommand{\intgb}{\int_{\mbox{\footnotesize ${\Gamma}$}}}
\newcommand{\intv}{\int_{\mbox{\footnotesize ${\sigma}$}}}
\newcommand{\sumv}{\sum_{K=1}^{N_{\mathrm{lev}}}}
\newcommand{\sumk}{\sum_{L=1}^{K}}
\newcommand{\sumN}{\sum_{i=1}^{N+1}}
\newcommand{\half}{\frac{1}{2}}
\newcommand{\inti}{\int_{\mbox{\footnotesize\sf I}}}
\newcommand{\intbd}{\oint_{\mbox{\footnotesize ${\delta}$\sf D}}}
\newcommand{\intbi}{\oint_{\mbox{\footnotesize ${\delta}$\sf I}}}
\newcommand{\ldnorm}[1]{\left\| #1 \right\|_{\mbox{\footnotesize \sf D}} }
\newcommand{\lonorm}[1]{\left\| #1 \right\|_{\Omega}}
\newcommand{\spc}[1]{\mbox{\sf #1}}
\newcommand{\ope}[1]{{\cal #1}}
\newcommand{\mt}[1]{{\rm #1}}
\newcommand{\dis}{\displaystyle}
\newcommand{\ve}{\varepsilon}
\newcommand{\ov}{\overline}
\newcommand{\wt}{\widetilde}
\newcommand{\wh}{\widehat}
\newcommand{\Dhat}{\widehat{D}}
\newcommand{\be}{\begin{equation}}
\newcommand{\ee}{\end{equation}}
\newcommand{\bea}{\begin{eqnarray*}}
\newcommand{\eea}{\end{eqnarray*}}
\newcommand{\Jace}{J^{(e)}}
\newcommand{\Jacl}{J^{(l)}}
\def\bepsilon{\mbox{\boldmath $\epsilon $}}
\def\bpsi{\mbox{\boldmath $\psi $}}
\def\bphi{\mbox{\boldmath $\phi $}}
\def\bmu{\mbox{\boldmath $\mu $}}
\def\Et{ \tilde{E} }
\def\Ht{ \tilde{H} }
\def\sdot{ \dot{\sigma} }

\newcommand{\fstar}{f^{(*)}}

\DeclareMathOperator{\Span}{span}
\DeclareMathOperator{\Dim}{dim}

\newcommand{\polyquad}{\mathcal{Q}_{N}}
\newcommand{\polyP}{\mathcal{P}_{N}}
\newcommand{\polyPnpm}{\mathcal{P}_{(N+M)}}
\newcommand{\polyPd}{\mathcal{P}_{d}}
\newcommand{\polyPnm}{\mathcal{P}_{N,M}}
\newcommand{\polyPn}{\mathcal{P}_{N,0}}
\newcommand{\transpose}{^{\mathcal{T}}}

\newcommand{\vecQ}{\vec{Q}}
\newcommand{\vecQe}{\vec{Q}^{(e)}}
\newcommand{\vecFe}{\vec{\mathcal{F}}^{(e)}}
\newcommand{\statevec}{\vec{Y}}
\newcommand{\statevecN}{\vec{Y}_N^{(e)}}
\newcommand{\statestage}{\vec{\mathcal{Y}}}
\newcommand{\Ftensor}{\vec{F}(\qvector)}
\newcommand{\FtensorN}{\vec{F}\left( \qvectorN \right)}
\newcommand{\FtensorStar}{\vec{F}\left( \qvector_N^{(e,k)} \right)}
\newcommand{\Svector}{S(\qvector)}
\newcommand{\SvectorN}{S \left( \qvectorN \right)}
\newcommand{\qref}{\vec{q}_0}
\newcommand{\qvectorb}{\vec{q}_b}
\newcommand{\qtt}{\vec{q}_{tt}}
\newcommand{\qhat}{\widehat{\vec{q}}}
\newcommand{\qhatb}{\widehat{\vec{q}}_b}
\newcommand{\qelem}{q^{(e)}}
\newcommand{\rhoref}{\rho_0}
\newcommand{\piref}{\pi_0}
\newcommand{\Thetaref}{\Theta_0}
\newcommand{\Gref}{G_0}
\newcommand{\Tref}{T_0}
\newcommand{\thetaref}{\theta_0}
\newcommand{\Pref}{{P}_0}
\newcommand{\Eref}{{E}_0}
\newcommand{\Href}{{h}_0}
\newcommand{\rhohat}{\widehat{\rho}}
\newcommand{\pihat}{\widehat{\pi}}
\newcommand{\Phat}{\widehat{P}}
\newcommand{\uvechat}{\widehat{{\mbox{\boldmath$u$\unboldmath}}}}
\newcommand{\uhathat}{\widehat{\widehat{{\mbox{\boldmath$u$\unboldmath}}}}}
\newcommand{\Uhat}{\widehat{{\mbox{\boldmath$U$\unboldmath}}}}
\newcommand{\Uhathat}{\widehat{\widehat{{\mbox{\boldmath$U$\unboldmath}}}}}
\newcommand{\thetahat}{\widehat{\theta}}
\newcommand{\Thetahat}{\widehat{\Theta}}
\newcommand{\Ehat}{\widehat{E}}
\newcommand{\uhat}{\widehat{u}}
\newcommand{\vhat}{\widehat{v}}
\newcommand{\what}{\widehat{w}}
\newcommand{\pitt}{\pi_{tt}}
\newcommand{\rhott}{\rho_{tt}}
\newcommand{\Ett}{E_{tt}}
\newcommand{\Utt}{\vec{U}_{tt}}
\newcommand{\uvectt}{\vec{u}_{tt}}
\newcommand{\utt}{u_{tt}}
\newcommand{\vtt}{v_{tt}}
\newcommand{\wtt}{w_{tt}}
\newcommand{\Ptt}{P_{tt}}
\newcommand{\vecPtt}{\vec{P}_{tt}}
\newcommand{\Thetatt}{\Theta_{tt}}
\newcommand{\thetatt}{\theta_{tt}}
%Projector Matrices
\newcommand{\projmatrix}{\vec{\mathcal{P}}}
\newcommand{\qmatrix}{\vec{\mathcal{Q}}}
\newcommand{\pcmatrix}{\vec{\mathcal{P}}_C}
\newcommand{\Cmatrix}{\left(\vec{\mathcal{C}}^{(e,f)}\right)\transpose}
\newcommand{\Dmatrix}{\vec{D}^{(e)}}
\newcommand{\Dwmatrix}{\wt{\vec{D}}^{(e)}}
\newcommand{\Mmatrix}{M^{(e)}}
\newcommand{\Fmatrix}{\vec{F}^{(e,l)}}
\newcommand{\Gmatrix}{\mathcal{G}}
\newcommand{\Umatrix}{\mathcal{U}^{(e,f)}}
\newcommand{\amatrix}{\vec{\mathcal{A}}}
\newcommand{\rmatrix}{\vec{\mathcal{R}}}
%Vectors
\newcommand{\nvector}{\wh{\vec{n}}_{\Gamma}}
\newcommand{\nhat}{\wh{\vec{n}}}
\newcommand{\ivector}{\wh{\vec{i}}}
\newcommand{\jvector}{\wh{\vec{j}}}
\newcommand{\kvector}{\wh{\vec{k}}}
\newcommand{\rvector}{\wh{\vec{r}}}
\newcommand{\svector}{\wh{\vec{s}}}
\newcommand{\tvector}{\wh{\vec{t}}}
\newcommand{\vvector}{\wh{\vec{v}}}
\newcommand{\Qvector}{\vec{Q}}
%Vectors
\newcommand{\ur}{{u}^{(r)}}
\newcommand{\us}{{u}^{(s)}}
\newcommand{\ut}{{u}^{(t)}}
\newcommand{\urtt}{{u}_{tt}^{(r)}}
\newcommand{\ustt}{{u}_{tt}^{(s)}}
\newcommand{\uttt}{{u}_{tt}^{(t)}}
\newcommand{\urhat}{\widehat{u}^{(r)}}
\newcommand{\ushat}{\widehat{u}^{(s)}}
\newcommand{\uthat}{\widehat{u}^{(t)}}
%Other Operators
\newcommand{\grad}{\vec{\nabla}}
\newcommand{\Grad}{\vec{\nabla}}
\newcommand{\Dskew}{\mathcal{D}}

\def\bepsilon{\mbox{\boldmath $\epsilon $}}
\def\bpsi{\mbox{\boldmath $\psi $}}
\def\bphi{\mbox{\boldmath $\phi $}}
\def\bmu{\mbox{\boldmath $\mu $}}
\def\Et{ \tilde{E} }
\def\Ht{ \tilde{H} }
\def\sdot{ \dot{\sigma} }
%\renewcommand{\thetable}{\Roman{table}}
%\renewcommand{\thefigure}{\arabic{figure}}

%\DeclareMathOperator{\Span}{span}
%\DeclareMathOperator{\Dim}{dim}

%Editing Commands
\newcommand{\here}{ \textcolor{red}{YOU ARE HERE}}

%Time-Integration
\newcommand{\dt}{{\Delta t}}
\newcommand\ST{\rule[-0.75em]{0pt}{2em}}
\newcommand{\Sfunction}{\mathcal{S}}
\newcommand{\Lfunction}{\mathcal{L}}
\newcommand{\Nfunction}{\mathcal{N}}

%DG Operators
\newcommand{\average}[1]{ \left\{ #1 \right\} }
\newcommand{\jump}[1]{ \llbracket #1 \rrbracket }

%HDG Matrices
\newcommand{\CCmatrix}{\mathcal{C}^{(e,k)}}
\newcommand{\Jmatrix}{\mathcal{J}^{(e,k)}}
\newcommand{\DDmatrix}{\wt{D}^{(e)}}
\newcommand{\SSvector}{\mathcal{S}(q)}
\newcommand{\cghdg}{cg\underline{\hspace{0.15cm}}to\underline{\hspace{0.15cm}}hdg}
%\newcommand{\ul}{\underline{\hspace{0.15cm}}}
\newcommand{\RRmatrix}{\mathcal{R}}

%Clima specific variables
\newcommand{\etotal}{e^{\mathrm{tot}}}
\newcommand{\Etotal}{E^{\mathrm{tot}}}
\newcommand{\Fvector}{\vec{\mathcal{F}}}
\newcommand{\Pvector}{\vec{\mathcal{P}}}
\newcommand{\Fadv}{\vec{\mathcal{F}}^{\mathrm{adv}}}
\newcommand{\Fndf}{\vec{\mathcal{F}}^{\mathrm{ndf}}}
\newcommand{\Fdiff}{\vec{\mathcal{F}}^{\mathrm{diff}}}
\newcommand{\Tvector}{\vec{\mathcal{T}}}
\newcommand{\Source}{\vec{\mathcal{S}}}

\newcommand{\fxg}[1]{\textcolor{cyan}{FXG: #1}}


\usepackage[inline]{enumitem}


\title{Notes on Single- and Multi-rate Runge-Kutta methods}

\begin{document}

\maketitle
\tableofcontents

\hl{[Thomas]: These notes will eventually be merged into the main numerics file.
	While these notes are being developed, let's keep them separate until we know
	how these notes fit into the main text.}

\section{Introduction}

\subsection{Overview of Basic Runge-Kutta Methods} To ensure a strong foundation, we start
with a brief overview of Runge-Kutta methods, with a particular emphasis
on the algorithmic aspects of the method. Suppose we wish to solve the following
ordinary differential equation (ODE):
\begin{equation}\label{eq:generic-ODE}
	\ddt{\vec{y}} = \vec{\mathcal{T}}(\vec{y}), \quad \vec{y}(t_0) = \vec{y}_0,
\end{equation}
where $\vec{y}(t)$ is the state vector, $\vec{\mathcal{T}}(\vec{y}(t)) = \vec{\mathcal{T}}(t, \vec{y})$
is the total tendency, and $\vec{y}(t_0) = \vec{y}_0$ describes an abstract initial-value condition.

\subsubsection{General Runge-Kutta methods}
A single step of an $s$-stage Runge-Kutta (RK) method for
solving \eqref{eq:generic-ODE} can be expressed as the following:
\begin{equation}\label{eq:gen-rk}
	\vec{y}^{n+1} = \vec{y}^n + \Delta t \sum_{i=1}^{s} b_i \vec{\mathcal{T}}(\vec{Y}^i),
\end{equation}
where $\vec{\mathcal{T}}(\vec{Y}^i)$ is the evaluation of the right-hand side tendency at the stage value $\vec{Y}^i$,
defined at each stage of the RK method:
\begin{equation}\label{eq:gen-rk-stage-vals}
	\vec{Y}^i := \vec{y}^{n} + \Delta t \sum_{j=1}^{s} a_{i,j} \vec{\mathcal{T}}(\vec{Y}^j).
\end{equation}
At the start of the RK method, the stage values are initialized as $\vec{Y}^{0} \leftarrow \vec{y}^n$
and the process above is executed overall all stages until $\vec{Y}^s$ is computed as the approximation
to $\vec{y}^{n+1}$. Mathematically, the stage vectors $\vec{Y}^i$ are best interpreted as
approximations to the state at the ``stage-times'' $t_n + c_i\Delta t$:
\begin{equation}
	\vec{Y}^i \approx \vec{y}(t_n + c_i\Delta t),
\end{equation}
where $c_i$ denote scaling factors which helps determine a point in the interval $\lbrack t_n, t_{n+1}\rbrack$. That
is, the coefficients determine the stage time: $t_n + c_i\Delta t \in \lbrack t_n, t_{n+1}\rbrack$.

In the above expressions, we define $\vec{A} = \lbrace a_{i,j} \rbrace \in \mathbb{R}^{s\times s}$,
$\vec{b} = \lbrace b_i \rbrace \in \mathbb{R}^s$, and
$\vec{c} = \lbrace c_i \rbrace \in \mathbb{R}^s$ as the characteristic coefficients of
a given RK method. This means we can associate any RK method with its so-called \textbf{Butcher Tableau}:
\begin{equation}\label{eq:gen-bt}
\begin{array}{c|c}
\ST \vec{c} &\vec{A}\\
\hline
\ST  & \vec{b}^T
\end{array} =
\begin{array}{c|c c c c}
\ST c_1 & a_{1,1} & a_{1,2} & \cdots & a_{1,s}\\
\ST c_2 & a_{2,1} & a_{2,2} & \cdots & a_{2,s}\\
\ST \vdots & \vdots & \vdots & \ddots & \vdots \\
\ST c_s & a_{s,1} & a_{s,2} & \cdots & a_{s,s}\\
\hline
\ST  & b_1 & b_2 & \cdots & b_s
\end{array}.
\end{equation}
The vector $\vec{c}$ is often called the \emph{consistency vector}, and is typically subject
to the row-sum condition:
\begin{equation}
c_i = \sum_{j=1}^{s} a_{i,j}, \quad \forall i = 1, \cdots, s.
\end{equation}
This simplifies the order conditions for higher-order RK methods. For more information
on general RK methods, we refer the interested reader to \cite[Ch. 5.2]{atkinson2011numerical}.

\subsubsection{Embedded Runge-Kutta methods} Embedded RK (ERK) methods are designed to readily provide
an estimate of the local truncation error of a single RK step. As a result, this
can be used to allow more fine-grain control of the error with \textbf{adaptive time-step sizes}.
This is accomplished using a second approximation $\widehat{\vec{y}}^{n+1}$ using the \emph{same}
stage values $\vec{Y}^i$ and coefficients $\vec{A}$ as in \eqref{eq:gen-bt}.
The second approximation is defined using a second set of coefficients $\widehat{\vec{b}} =
\lbrace \widehat{b}_i, i = 1, \cdots, s\rbrace$, with
\begin{equation}
	\widehat{\vec{y}}^{n+1} = \vec{y}^n + \Delta t \sum_{i=1}^{s} \widehat{b}_i \vec{\mathcal{T}}(\vec{Y}^i).
\end{equation}
Once both $\vec{y}^{n+1}$ and $\widehat{\vec{y}}^{n+1}$
are determined, an approximation to the truncation error is arrive at via:
\begin{equation}
	\vec{e}^{n+1} := \vec{y}^{n+1} - \widehat{\vec{y}}^{n+1} =
	\Delta t \sum_{i=1}^{s}\left(b_i - \widehat{b}_i\right)\vec{\mathcal{T}}(\vec{Y}^i).
\end{equation}
For ERK methods, the coefficients are summarized in the following expanded Butcher Tableau:
\begin{equation}\label{eq:gen-bt-form}
\begin{array}{c|c}
\ST \vec{c} &\vec{A}\\
\hline
\ST  & \vec{b}^T \\
\hline
\ST  & \widehat{\vec{b}}^T
\end{array} =
\begin{array}{c|c c c c}
\ST c_1 & a_{1,1} & a_{1,2} & \cdots & a_{1,s}\\
\ST c_2 & a_{2,1} & a_{2,2} & \cdots & a_{2,s}\\
\ST \vdots & \vdots & \vdots & \ddots & \vdots \\
\ST c_s & a_{s,1} & a_{s,2} & \cdots & a_{s,s}\\
\hline
\ST  & b_1 & b_2 & \cdots & b_s \\
\hline
\ST  & \widehat{b}_1 & \widehat{b}_2 & \cdots & \widehat{b}_s
\end{array}.
\end{equation}
Examples of ERK methods include the Runge-Kutta-Fehlberg \cite{fehlberg1969low}
and the Bogacki-Shampine \cite{bogacki19893} methods. ERK methods can be explicit
or implicit.

\paragraph{Explicit RK methods.} Whenever an RK method is fully explicit,
then all stage evaluations of the tendency
only depend on its previous evaluations. Within Butcher tableau notation, this
produces a coefficient matrix $\vec{A}$ which is strictly lower-triangular:
\begin{equation}
	\vec{A} = \begin{bmatrix}
	0        & 0        & 0      & \cdots     & 0 \\
	a_{2, 1} & 0        & 0      & \cdots     & 0 \\
	a_{3, 1} & a_{3, 2} & 0      & \cdots     & 0 \\
	\vdots   &          & \ddots & \ddots     & \vdots \\
	a_{s, 1} & a_{s, 2} & \cdots & a_{s, s-1} & 0
	\end{bmatrix}.
\end{equation}
As a result, $c_1 = 0$ clearly. Due to the lower-triangular structure, an explicit RK
method will have stage evaluations running from $j$ to $i-1$ (rather than $s$ as shown
for the general case in \eqref{eq:gen-rk-stage-vals}).

\paragraph{Implicit RK methods.} A general implicit RK method will have a
full Butcher Tableau matrix $A$ (as shown in \eqref{eq:gen-bt})
and stage evaluations of the tendencies require significantly more
computation per step size. However, they are far more stable for stiff differential equations;
this means that implicit methods often are able to take substantially larger
time-step sizes $\Delta t$ versus explicit RK methods.

Among the most popular implicit RK methods are known as \textbf{Explicit first step, Single
diagonal coefficient, Diagonally Implicit Runge-Kutta (ESDIRK)} methods.
The Butcher Tableau for such methods typically take the form
\begin{equation}\label{eq:esdirk-bt}
\begin{array}{c|c c c c c}
\ST 0 & 0 & 0 & 0 & \cdots & 0 \\
\ST 2\gamma & \gamma & \gamma & 0 & \cdots & 0\\
\ST c_3 & a_{3, 1} & a_{3, 2} & \gamma & \cdots & 0 \\
\ST \vdots & \vdots & \vdots & \vdots & \ddots & \vdots \\
\ST 1 & a_{s,1} & a_{s,2} & a_{s,3} & \cdots & \gamma \\
\hline
\ST  & b_1 & b_2 & b_3 & \cdots & \gamma \\
\hline
\ST  & \widehat{b}_1 & \widehat{b}_2 & \widehat{b}_3 & \cdots & \widehat{b}_s
\end{array}.
\end{equation}

For the remainder of this section, we shall associate a given RK (explicit, implicit, embedded) method with its
\emph{generalized Butcher Tableau} of the form \eqref{eq:gen-bt-form}.

\subsection{Additively-Partitioned Runge-Kutta (ARK) Methods}
Additively-partitioned RK methods allow for more specialized treatment of
specific tendency terms. For a more general case, suppose \eqref{eq:generic-ODE}
can be written as:
\begin{equation}\label{eq:additiveode}
	\ddt{\vec{y}} = \vec{\mathcal{T}}(\vec{y}) =
	\sum_{\nu = 1}^{N} \vec{\mathcal{T}}^{\lbrack \nu\rbrack}(\vec{y}),
\end{equation}
where $\vec{\mathcal{T}}^{\lbrack \nu\rbrack}$ denote certain components of the right-hand side
tendency, whose distinctive properties warrant separate consideration. In general, a single step of
an $s$-stage, $N$-part additive RK method ($ARK_N$) 
is defined by its generalized Butcher tableau:
\begin{equation}\label{eq:ARKN-BT}
\begin{array}{c|c| c | c}
\ST \vec{c} &\vec{A}_{1} & \cdots & \vec{A}_{N}\\
\hline
\ST  & \vec{b}_1^T & \cdots & \vec{b}_N^T\\
\hline
\ST  & \widehat{\vec{b}}_1^T & \cdots & \widehat{\vec{b}}_N^T
\end{array} =
\begin{array}{c|c c c | c | c c c }
\ST c_1 & a^{\lbrack 1 \rbrack}_{1,1} & \cdots & a^{\lbrack 1 \rbrack}_{1,s} & \cdots
& a^{\lbrack \nu \rbrack}_{1,1} & \cdots & a^{\lbrack \nu \rbrack}_{1,s}\\
\ST \vdots & \vdots & \ddots & \vdots & \cdots
& \vdots & \ddots & \vdots \\
\ST c_s & a^{\lbrack 1 \rbrack}_{s,1} & \cdots & a^{\lbrack 1 \rbrack}_{s,s} & \cdots
& a^{\lbrack \nu \rbrack}_{s,1} & \cdots & a^{\lbrack \nu \rbrack}_{s,s}\\
\hline
\ST  & b^{\lbrack 1 \rbrack}_1 & \cdots & b^{\lbrack 1 \rbrack}_s & \cdots
& b^{\lbrack \nu \rbrack}_1 & \cdots & b^{\lbrack \nu \rbrack}_s\\
\hline
\ST  & \widehat{b}^{\lbrack 1 \rbrack}_1 & \cdots & \widehat{b}^{\lbrack 1 \rbrack}_s &
& \widehat{b}^{\lbrack \nu \rbrack}_1 & \cdots & \widehat{b}^{\lbrack \nu \rbrack}_s
\end{array}
\end{equation}
and is given by
\begin{equation}
	\vec{y}^{n+1} = \vec{y}^n + \Delta t\left(
	\underbrace{\sum_{i=1}^{s}}_{\text{Stages}}
	\underbrace{\sum_{\nu=1}^{N}}_{\text{Components}}
	b_i^{\lbrack \nu \rbrack} \vec{\mathcal{T}}^{\lbrack \nu\rbrack}(\vec{Y}^i),
	\right)
\end{equation}
where the stage values are given by:
\begin{equation}
	\vec{Y}^i = \vec{y}^n + \Delta t \sum_{j=1}^{s} \sum_{\nu = 1}^{N} a_{i,j}^{\lbrack \nu \rbrack}
	\vec{\mathcal{T}}^{\lbrack \nu\rbrack}(\vec{Y}^j).
\end{equation}
Similar to standard RK methods, the stage vectors are approximations to the state at each stage
of the $ARK_N$ method. Moreover, the temporal coefficients $c_i$ satisfy a similar
row-sum condition, holding for all $\nu = 1, \cdots, N$:
\begin{equation}
	c_i = \sum_{j=1}^{s} a_{i, j}^{\lbrack \nu \rbrack}, \quad \forall \nu = 1, \cdots, N.
\end{equation}
The Butcher coefficients $\vec{c}$, $\vec{b}_{\nu}$, $\vec{A}_{\nu}$, and $\widehat{\vec{b}}_{\nu}$
are constrained by certain accuracy and stability requirements, which are summarized in
\cite{kennedy2001additive}.

\subsubsection{Implicit-Explicit (IMEX) Runge-Kutta methods}
A common setting is the case $N = 2$. This gives the typical context for
Implicit-Explicit (IMEX) splitting methods, where the tendency $\vec{\mathcal{T}}$
is assumed to have the decomposition:
\begin{equation}
	\ddt{\vec{y}} =
	\vec{\mathcal{T}}_{s}(\vec{y}) + \vec{\mathcal{T}}_{ns}(\vec{y}),
\end{equation}
where the right-hand side has been split into a ``stiff'' component $\vec{\mathcal{T}}_{s}$,
to be treated implicitly, and a non-stiff part $\vec{\mathcal{T}}_{ns}$ to be treated explicitly.
Two different RK methods are applied to $\vec{\mathcal{T}}_{s}$ and $\vec{\mathcal{T}}_{ns}$
separately, which have been specifically designed and coupled. Examples can be found in
\cite{giraldo:2013}. The Butcher Tableau for an $ARK_2$ method will have the
form
\begin{equation}\label{eq:imex-bt}
	\begin{array}{c|c|c}
	\ST \vec{c} &\vec{A}_E &\vec{A}_I\\
	\hline
	\ST  & \vec{b}_E^T & \vec{b}_I^T \\
	\hline
	\ST  & \widehat{\vec{b}}_E^T & \widehat{\vec{b}}_I^T
	\end{array},
\end{equation}
with
\begin{equation}
	\vec{A}_O = \left\lbrace a_{i, j}^O \right\rbrace, \quad
	\vec{b}_O = \left\lbrace b_{i}^O \right\rbrace, \quad
	\widehat{\vec{b}}_O = \left\lbrace \widehat{b}_{i}^O \right\rbrace,
\end{equation}
where $O$ denotes the label (either $E$ for explicit or $I$ for implicit).
One step of an $s$-stage $ARK_2$ method defined by the Butcher tableau \eqref{eq:imex-bt}
is given by
\begin{equation}
	\vec{y}^{n+1} = \vec{y}^n + \Delta t \sum_{i=1}^{s}\left(
	b_i^E \vec{\mathcal{T}}_{ns}(\vec{Y}^i) + b_i^I \vec{\mathcal{T}}_{s}(\vec{Y}^i)
	\right),
\end{equation}
with stage values defined as
\begin{equation}
	\vec{Y}^i = \vec{y}^n + \Delta t \sum_{j=1}^{s}
	\left(
	a^E_{i,j} \vec{\mathcal{T}}_{ns}(\vec{Y}^j) +
	a^I_{i,j} \vec{\mathcal{T}}_{s}(\vec{Y}^j)
	\right).
\end{equation}
Restricting our attention on $ARK_2$ schemes using a diagonally
implicit RK scheme (upper-triangular part of $\vec{A}_I$ is zero, see for example \cite{giraldo:2013}),
we can rearrange the expression for $\vec{Y}^i$ as:
\begin{equation}
	\vec{Y}^i = \vec{y}^n + \Delta t \sum_{j=1}^{i-1}\left(
	a^E_{i,j} \vec{\mathcal{T}}_{ns}(\vec{Y}^j) +
	a^I_{i,j} \vec{\mathcal{T}}_{s}(\vec{Y}^j)
	\right)
	+ \Delta t a^I_{i,i} \vec{\mathcal{T}}_{s}(\vec{Y}^i).
\end{equation}
After another rearranging, we obtain the implicit equation
for the stage vector $\vec{Y}^i$:
\begin{equation}\label{eq:implicit-equation-IMEX}
	\left( \vec{I} - \Delta t a_{i,i}^I \vec{\mathcal{T}}_{s}\right)\vec{Y}^i
	= \vec{y}^n + \Delta t \sum_{j=1}^{i-1}\left(
	a^E_{i,j} \vec{\mathcal{T}}_{ns}(\vec{Y}^j) +
	a^I_{i,j} \vec{\mathcal{T}}_{s}(\vec{Y}^j)
	\right).
\end{equation}

In general, the left-hand side operator of \eqref{eq:implicit-equation-IMEX}
may be \emph{nonlinear}, in which case, a nonlinear solver will need to be
employed. In the case that $\vec{\mathcal{T}}_{s}$ is \emph{linear},
then $\vec{\mathcal{T}}_{s}(\vec{Y}^i) = \vec{\mathcal{L}}\vec{Y}^i$, with
$\vec{\mathcal{L}}$ a linear operator. Then at each stage of the ARK method,
we require the solution to the discrete linear system:
\begin{equation}
	\left( \vec{I} - \Delta t a_{i,i}^I \vec{\mathcal{L}}\right)\vec{Y}^i =
	\widehat{\vec{Y}}^i,
\end{equation}
where
\begin{equation}
	\widehat{\vec{Y}}^i = \vec{y}^n + \Delta t \sum_{j=1}^{i-1}\left(
	a^E_{i,j} \vec{\mathcal{T}}_{ns}(\vec{Y}^j) +
	a^I_{i,j} \vec{\mathcal{T}}_{s}(\vec{Y}^j)
	\right).
\end{equation}

If available, an error estimator can be constructed using the embedded scheme:
\begin{equation}
	\widehat{\vec{y}}^{n+1} = \vec{y}^n + \Delta t \sum_{i=1}^{s}\left(
	\widehat{b}_i^E \vec{\mathcal{T}}_{ns}(\vec{Y}^i) + \widehat{b}_i^I \vec{\mathcal{T}}_{s}(\vec{Y}^i)
	\right),
\end{equation}
which can be used in the error estimate: $\vec{e}^{n+1} = \vec{y}^{n+1} - \widehat{\vec{y}}^{n+1}$.

\subsection{Generalized Additively-Partitioned Runge-Kutta (GARK) Methods}
The GARK family of RK methods extends the traditional approach of ARK methods
by allowing for different stage values for the different components of the right-hand side.
Following \cite{sandu2015generalized}, suppose the ODE in question has
the form
\begin{equation}\label{eq:gark-ode}
\ddt{\vec{y}} = \sum_{\nu = 1}^{N} \vec{\mathcal{T}}^{\lbrack \nu\rbrack}(\vec{y}).
\end{equation}
Then one step of a GARK scheme reads as follows:
\begin{equation}
	\vec{y}^{n+1} = \vec{y}^n + \Delta t \left(
	\underbrace{\sum_{\nu = 1}^{N}}_{Components}
	\underbrace{\sum_{i = 1}^{s(\nu)}}_{Stages}
	b_i^{\lbrack \nu\rbrack} \vec{\mathcal{T}}^{\lbrack \nu\rbrack}(\vec{Y}^i_{\nu})
	\right),
\end{equation}
where the stage values $\vec{Y}^i_{\nu}$ are determined via:
\begin{equation}
	\vec{Y}^i_{\nu} = \vec{y}^n + \Delta t \sum_{m=1}^{N} \sum_{j=1}^{s(m)} a_{i, j}^{\lbrack \nu,m\rbrack}
	\vec{\mathcal{T}}^{\lbrack m\rbrack}(\vec{Y}^j_{m}).
\end{equation}
The corresponding Butcher tableau now has the block-wise form
\begin{equation}
	\begin{array}{c|c|c|c|c}
	\ST \vec{c}_{1} &\vec{A}_{1,1} &\vec{A}_{1,2} & \cdots & \vec{A}_{1,N}\\
	\ST \vec{c}_{2} &\vec{A}_{2,1} &\vec{A}_{2,2} & \cdots & \vec{A}_{2,N}\\
	\ST \vdots & \vdots & \vdots & \ddots & \vdots \\
	\ST \vec{c}_N & \vec{A}_{N,1} &\vec{A}_{N,2} & \cdots & \vec{A}_{N,N}\\
	\hline
	\ST  & \vec{b}_{1}^T & \vec{b}_{2}^T & \cdots & \vec{b}_{N}^T
	\end{array}
\end{equation}
where
\begin{equation}
	\vec{A}_{\sigma, \nu} = \left\lbrace a_{i, j}^{\lbrack \sigma, \nu \rbrack} \right\rbrace, \quad
	\vec{b}_{\nu} = \left\lbrace b_i^{\lbrack \nu \rbrack}\right\rbrace.
\end{equation}
The coefficient arrays $\vec{A}_{\nu, \nu}$ and $\vec{b}_{\nu}$ are best interpreted as stand-alone
RK methods applied to the $\nu$-th component of \eqref{eq:gark-ode}.
The consistency arrays $\vec{c}_m = \lbrace c_i^{\lbrack m \rbrack}\rbrace$
must adhere to a so-called \emph{internally consistent GARK condition},
namely:
\begin{equation}
	c_i^{\lbrack m \rbrack} = \sum_{j=1}^{s(1)} a_{i, j}^{\lbrack m, 1 \rbrack} =
	\sum_{j=1}^{s(2)} a_{i, j}^{\lbrack m, 2 \rbrack} = \cdots =
	\sum_{j=1}^{s(N)} a_{i, j}^{\lbrack m, N \rbrack},
\end{equation}
for $i = 1, \cdots, s(m)$, $m = 1, \cdots, N$.

The off-diagonal matrices $\vec{A}_{\sigma, \nu}$, $\sigma \neq \nu$ are generally non-square
and should be regarded as a coupling mechanism among the various components, i.e.
$\vec{A}_{\sigma, \nu}$ couples together the $\sigma$- and $\nu$-components together when assembling
the full approximation.

\hl{[Thomas]: TODO: expand this once we finish GARK details.}

\section{Multirate Runge-Kutta Methods}

\subsection{An explicit two-rate example}

Suppose we have an ODE of the form:
\begin{equation}
	\ddt{\vec{y}} = \vec{\mathcal{S}}(\vec{y}) + \vec{\mathcal{F}}(\vec{y}), \quad \vec{y}(t_0) = \vec{y}_0,
\end{equation}
where $\vec{\mathcal{S}}$ denotes a ``slow'' tendency and $\vec{\mathcal{F}}$ is the ``fast'' (stiff) tendency.
Then, for a given time-step size $\Delta t$, the two-rate method in \cite{Schlegel_2009} is summarized
as the following:
\begin{align}
	\vec{Y}_1 &= \vec{y}(t_n), \\
	\vec{r}_{i} &= \sum_{j=1}^{i-1}\tilde{a}^O_{ij}\vec{\mathcal{S}}(\vec{Y}_{j}), \\
	\vec{w}_{i,1} &= \vec{Y}_{i-1},\\
	\vec{w}_{i,k} &= \vec{w}_{i,k-1} + \Delta t \tilde{c}_i^O \sum_{j=1}^{k-1}\tilde{a}^I_{k,j}
	\left(\frac{1}{\tilde{c}_i^O}\vec{r}_i + \vec{\mathcal{F}}(\vec{w}_{i,j})\right),\\
	& \quad\quad i = 2, \cdots, s^O + 1 \text{ and } k = 2, \cdots, s^I + 1,\nonumber \\
	\vec{Y}_i &= \vec{w}_{i,s^I + 1}
\end{align}
where the tilde parameters denote increments per RK stage:
\begin{align}
	\tilde{a}_{ij} &= \begin{cases}
		a_{i,j} - a_{i-1, j} & \text{if } i < s + 1 \\
		b_j - a_{s,j} & \text{if } i = s + 1
	\end{cases},\\
	\tilde{c}_{i} &= \begin{cases}
		c_{i} - c_{i-1} & \text{if } i < s + 1 \\
		1 - c_{s} & \text{if } i = s + 1
	\end{cases},
\end{align}
where the coefficients $a$, $b$, and $c$ correspond to the Butcher tableau for a given RK method.
The superscripts ``$O$'' and ``$I$'' denote the outer (slow) and inner (fast) components of the multirate method
respectively. Thus, tilde coefficients should be associated with the RK method indicated by the superscripts.
In other words, the RK methods for the slow $\vec{\mathcal{S}}$ and fast $\vec{\mathcal{F}}$ components
have Butcher tables given by:
\begin{equation}
	\begin{array}{c|c}
	\ST \vec{c}_{O} &\vec{A}_{O} \\
	\hline
	\ST  & \vec{b}_O^T
	\end{array}, \quad
	\begin{array}{c|c}
	\ST \vec{c}_{I} &\vec{A}_{I} \\
	\hline
	\ST  & \vec{b}_I^T
	\end{array},
\end{equation}
where $\vec{A}_O = \lbrace a_{i,j}^O\rbrace$, $\vec{b}_O = \lbrace b_i^O \rbrace$, and
$c_O = \lbrace c_i^O \rbrace$ (similarly for $\vec{A}_I$, $\vec{b}_I$, and $\vec{c}_I$).
The method described here is for an explicit RK outer method with $s$ stages.

\hl{[Thomas]: TODO: expand on details.}

\subsection{Multirate GARK}
\hl{[Thomas]: TODO: finish GARK details.}

%-------Bibliography
\bibliographystyle{plain}
\bibliography{Giraldo_refs}

\end{document}