\documentclass{report}

\usepackage[T1]{fontenc}
\usepackage[utf8]{inputenc}
\usepackage{times}

\usepackage[font=small,labelfont=bf,tableposition=top]{caption}
\usepackage{graphicx}
\usepackage{natbib} 

\usepackage{amsmath}
\usepackage{amsfonts}
\usepackage{amssymb}
\usepackage{color, soul}
\usepackage{hyperref}
\usepackage{algorithmicx}
\usepackage{algpseudocode}
\usepackage{subfigure}
\usepackage{stmaryrd}

\renewcommand{\vec}[1]{\boldsymbol{{#1}}} 
\newcommand{\duesoon}[1]{{\sethlcolor{green}\hl{#1}}}
\usepackage{mathrsfs}


\newtheorem{theorem}{Theorem}
\newtheorem{acknowledgement}[theorem]{Acknowledgement}
\newtheorem{algorithm}[theorem]{Algorithm}
\newtheorem{axiom}[theorem]{Axiom}
\newtheorem{case}[theorem]{Case}
\newtheorem{claim}[theorem]{Claim}
\newtheorem{conclusion}[theorem]{Conclusion}
\newtheorem{condition}[theorem]{Condition}
\newtheorem{conjecture}[theorem]{Conjecture}
\newtheorem{corollary}[theorem]{Corollary}
\newtheorem{criterion}[theorem]{Criterion}
\newtheorem{definition}[theorem]{Definition}
\newtheorem{example}[theorem]{Example}
\newtheorem{exercise}[theorem]{Exercise}
\newtheorem{lemma}[theorem]{Lemma}
\newtheorem{notation}[theorem]{Notation}
\newtheorem{problem}[theorem]{Problem}
\newtheorem{proposition}[theorem]{Proposition}
\newtheorem{remark}[theorem]{Remark}
\newtheorem{solution}[theorem]{Solution}
\newtheorem{summary}[theorem]{Summary}
\newenvironment{proof}[1][Proof]{\textbf{#1.} }{\ \rule{0.5em}{0.5em}}

\newtheorem{guess}{Definition}
\newcommand{\comment}[1] {}
\newcommand{\Norder} {N}
\newcommand{\order}{\mathcal{O}}
\newcommand{\Npoints} {N_p}
\newcommand{\Nfaces} {N_{f}}
\newcommand{\Nelements} {N_e}

\newcommand{\eps}{\varepsilon}
\newcommand{\Dweak}{\wt{D}}
\newcommand{\diff}[2] {\frac{\partial #1}{\partial #2}}
\newcommand{\dxx}[2] {\frac{\partial^2 #1}{\partial {#2}^2}}
\newcommand{\difft}[2] {\frac{d #1}{d #2}}
\newcommand{\dxxt}[2] {\frac{d^2 #1}{d {#2}^2}}
\newcommand{\lagrange}[1] {\frac{d #1}{dt}}
\newcommand{\lebesgue}{\parallel I \parallel}
\newcommand{\polysp}{\mathcal{P}_N}
\newcommand{\laplacian}{\nabla^2}
\newcommand{\divergence}{\nabla \cdot}
\newcommand{\inte}{\int_{\mbox{\footnotesize ${\Omega_e}$}}}
\newcommand{\intb}{\int_{\mbox{\footnotesize ${\Gamma_e}$}}}
\newcommand{\intce}{\int_{\mbox{\footnotesize ${\widehat{\Omega}_e}$}}}
\newcommand{\intcb}{\int_{\mbox{\footnotesize ${\widehat{\Gamma}_e}$}}}
\newcommand{\intg}{\int_{\mbox{\footnotesize ${\Omega}$}}}
\newcommand{\intgb}{\int_{\mbox{\footnotesize ${\Gamma}$}}}
\newcommand{\intv}{\int_{\mbox{\footnotesize ${\sigma}$}}}
\newcommand{\sumv}{\sum_{K=1}^{N_{\mathrm{lev}}}}
\newcommand{\sumk}{\sum_{L=1}^{K}}
\newcommand{\sumN}{\sum_{i=1}^{N+1}}
\newcommand{\half}{\frac{1}{2}}
\newcommand{\inti}{\int_{\mbox{\footnotesize\sf I}}}
\newcommand{\intbd}{\oint_{\mbox{\footnotesize ${\delta}$\sf D}}}
\newcommand{\intbi}{\oint_{\mbox{\footnotesize ${\delta}$\sf I}}}
\newcommand{\ldnorm}[1]{\left\| #1 \right\|_{\mbox{\footnotesize \sf D}} }
\newcommand{\lonorm}[1]{\left\| #1 \right\|_{\Omega}}
\newcommand{\spc}[1]{\mbox{\sf #1}}
\newcommand{\ope}[1]{{\cal #1}}
\newcommand{\mt}[1]{{\rm #1}}
\newcommand{\dis}{\displaystyle}
\newcommand{\ve}{\varepsilon}
\newcommand{\ov}{\overline}
\newcommand{\wt}{\widetilde}
\newcommand{\wh}{\widehat}
\newcommand{\Dhat}{\widehat{D}}
\newcommand{\be}{\begin{equation}}
\newcommand{\ee}{\end{equation}}
\newcommand{\bea}{\begin{eqnarray*}}
\newcommand{\eea}{\end{eqnarray*}}
\newcommand{\Jace}{J^{(e)}}
\newcommand{\Jacl}{J^{(l)}}
\def\bepsilon{\mbox{\boldmath $\epsilon $}}
\def\bpsi{\mbox{\boldmath $\psi $}}
\def\bphi{\mbox{\boldmath $\phi $}}
\def\bmu{\mbox{\boldmath $\mu $}}
\def\Et{ \tilde{E} }
\def\Ht{ \tilde{H} }
\def\sdot{ \dot{\sigma} }

\newcommand{\fstar}{f^{(*)}}

\DeclareMathOperator{\Span}{span}
\DeclareMathOperator{\Dim}{dim}

\newcommand{\polyquad}{\mathcal{Q}_{N}}
\newcommand{\polyP}{\mathcal{P}_{N}}
\newcommand{\polyPnpm}{\mathcal{P}_{(N+M)}}
\newcommand{\polyPd}{\mathcal{P}_{d}}
\newcommand{\polyPnm}{\mathcal{P}_{N,M}}
\newcommand{\polyPn}{\mathcal{P}_{N,0}}
\newcommand{\transpose}{^{\mathcal{T}}}

\newcommand{\vecQ}{\vec{Q}}
\newcommand{\vecQe}{\vec{Q}^{(e)}}
\newcommand{\vecFe}{\vec{\mathcal{F}}^{(e)}}
\newcommand{\statevec}{\vec{Y}}
\newcommand{\statevecN}{\vec{Y}_N^{(e)}}
\newcommand{\statestage}{\vec{\mathcal{Y}}}
\newcommand{\Ftensor}{\vec{F}(\qvector)}
\newcommand{\FtensorN}{\vec{F}\left( \qvectorN \right)}
\newcommand{\FtensorStar}{\vec{F}\left( \qvector_N^{(e,k)} \right)}
\newcommand{\Svector}{S(\qvector)}
\newcommand{\SvectorN}{S \left( \qvectorN \right)}
\newcommand{\qref}{\vec{q}_0}
\newcommand{\qvectorb}{\vec{q}_b}
\newcommand{\qtt}{\vec{q}_{tt}}
\newcommand{\qhat}{\widehat{\vec{q}}}
\newcommand{\qhatb}{\widehat{\vec{q}}_b}
\newcommand{\qelem}{q^{(e)}}
\newcommand{\rhoref}{\rho_0}
\newcommand{\piref}{\pi_0}
\newcommand{\Thetaref}{\Theta_0}
\newcommand{\Gref}{G_0}
\newcommand{\Tref}{T_0}
\newcommand{\thetaref}{\theta_0}
\newcommand{\Pref}{{P}_0}
\newcommand{\Eref}{{E}_0}
\newcommand{\Href}{{h}_0}
\newcommand{\rhohat}{\widehat{\rho}}
\newcommand{\pihat}{\widehat{\pi}}
\newcommand{\Phat}{\widehat{P}}
\newcommand{\uvechat}{\widehat{{\mbox{\boldmath$u$\unboldmath}}}}
\newcommand{\uhathat}{\widehat{\widehat{{\mbox{\boldmath$u$\unboldmath}}}}}
\newcommand{\Uhat}{\widehat{{\mbox{\boldmath$U$\unboldmath}}}}
\newcommand{\Uhathat}{\widehat{\widehat{{\mbox{\boldmath$U$\unboldmath}}}}}
\newcommand{\thetahat}{\widehat{\theta}}
\newcommand{\Thetahat}{\widehat{\Theta}}
\newcommand{\Ehat}{\widehat{E}}
\newcommand{\uhat}{\widehat{u}}
\newcommand{\vhat}{\widehat{v}}
\newcommand{\what}{\widehat{w}}
\newcommand{\pitt}{\pi_{tt}}
\newcommand{\rhott}{\rho_{tt}}
\newcommand{\Ett}{E_{tt}}
\newcommand{\Utt}{\vec{U}_{tt}}
\newcommand{\uvectt}{\vec{u}_{tt}}
\newcommand{\utt}{u_{tt}}
\newcommand{\vtt}{v_{tt}}
\newcommand{\wtt}{w_{tt}}
\newcommand{\Ptt}{P_{tt}}
\newcommand{\vecPtt}{\vec{P}_{tt}}
\newcommand{\Thetatt}{\Theta_{tt}}
\newcommand{\thetatt}{\theta_{tt}}
%Projector Matrices
\newcommand{\projmatrix}{\vec{\mathcal{P}}}
\newcommand{\qmatrix}{\vec{\mathcal{Q}}}
\newcommand{\pcmatrix}{\vec{\mathcal{P}}_C}
\newcommand{\Cmatrix}{\left(\vec{\mathcal{C}}^{(e,f)}\right)\transpose}
\newcommand{\Dmatrix}{\vec{D}^{(e)}}
\newcommand{\Dwmatrix}{\wt{\vec{D}}^{(e)}}
\newcommand{\Mmatrix}{M^{(e)}}
\newcommand{\Fmatrix}{\vec{F}^{(e,l)}}
\newcommand{\Gmatrix}{\mathcal{G}}
\newcommand{\Umatrix}{\mathcal{U}^{(e,f)}}
\newcommand{\amatrix}{\vec{\mathcal{A}}}
\newcommand{\rmatrix}{\vec{\mathcal{R}}}
%Vectors
\newcommand{\nvector}{\wh{\vec{n}}_{\Gamma}}
\newcommand{\nhat}{\wh{\vec{n}}}
\newcommand{\ivector}{\wh{\vec{i}}}
\newcommand{\jvector}{\wh{\vec{j}}}
\newcommand{\kvector}{\wh{\vec{k}}}
\newcommand{\rvector}{\wh{\vec{r}}}
\newcommand{\svector}{\wh{\vec{s}}}
\newcommand{\tvector}{\wh{\vec{t}}}
\newcommand{\vvector}{\wh{\vec{v}}}
\newcommand{\Qvector}{\vec{Q}}
%Vectors
\newcommand{\ur}{{u}^{(r)}}
\newcommand{\us}{{u}^{(s)}}
\newcommand{\ut}{{u}^{(t)}}
\newcommand{\urtt}{{u}_{tt}^{(r)}}
\newcommand{\ustt}{{u}_{tt}^{(s)}}
\newcommand{\uttt}{{u}_{tt}^{(t)}}
\newcommand{\urhat}{\widehat{u}^{(r)}}
\newcommand{\ushat}{\widehat{u}^{(s)}}
\newcommand{\uthat}{\widehat{u}^{(t)}}
%Other Operators
\newcommand{\grad}{\vec{\nabla}}
\newcommand{\Grad}{\vec{\nabla}}
\newcommand{\Dskew}{\mathcal{D}}

\def\bepsilon{\mbox{\boldmath $\epsilon $}}
\def\bpsi{\mbox{\boldmath $\psi $}}
\def\bphi{\mbox{\boldmath $\phi $}}
\def\bmu{\mbox{\boldmath $\mu $}}
\def\Et{ \tilde{E} }
\def\Ht{ \tilde{H} }
\def\sdot{ \dot{\sigma} }
%\renewcommand{\thetable}{\Roman{table}}
%\renewcommand{\thefigure}{\arabic{figure}}

%\DeclareMathOperator{\Span}{span}
%\DeclareMathOperator{\Dim}{dim}

%Editing Commands
\newcommand{\here}{ \textcolor{red}{YOU ARE HERE}}

%Time-Integration
\newcommand{\dt}{{\Delta t}}
\newcommand\ST{\rule[-0.75em]{0pt}{2em}}
\newcommand{\Sfunction}{\mathcal{S}}
\newcommand{\Lfunction}{\mathcal{L}}
\newcommand{\Nfunction}{\mathcal{N}}

%DG Operators
\newcommand{\average}[1]{ \left\{ #1 \right\} }
\newcommand{\jump}[1]{ \llbracket #1 \rrbracket }

%HDG Matrices
\newcommand{\CCmatrix}{\mathcal{C}^{(e,k)}}
\newcommand{\Jmatrix}{\mathcal{J}^{(e,k)}}
\newcommand{\DDmatrix}{\wt{D}^{(e)}}
\newcommand{\SSvector}{\mathcal{S}(q)}
\newcommand{\cghdg}{cg\underline{\hspace{0.15cm}}to\underline{\hspace{0.15cm}}hdg}
%\newcommand{\ul}{\underline{\hspace{0.15cm}}}
\newcommand{\RRmatrix}{\mathcal{R}}

%Clima specific variables
\newcommand{\etotal}{e^{\mathrm{tot}}}
\newcommand{\Etotal}{E^{\mathrm{tot}}}
\newcommand{\Fvector}{\vec{\mathcal{F}}}
\newcommand{\Pvector}{\vec{\mathcal{P}}}
\newcommand{\Fadv}{\vec{\mathcal{F}}^{\mathrm{adv}}}
\newcommand{\Fndf}{\vec{\mathcal{F}}^{\mathrm{ndf}}}
\newcommand{\Fdiff}{\vec{\mathcal{F}}^{\mathrm{diff}}}
\newcommand{\Tvector}{\vec{\mathcal{T}}}
\newcommand{\Source}{\vec{\mathcal{S}}}

\newcommand{\fxg}[1]{\textcolor{cyan}{FXG: #1}}


\usepackage[inline]{enumitem}
\usepackage{xcolor}


\title{CLIMA Cloud Microphysics} 
\author{ }

\begin{document}

\maketitle
\tableofcontents

\chapter{Liquid-Phase Microphysics Scheme}
In this section of the design documents we focus on warm rain microphysical processes related to condensation and evaporation, sedimentation, and collisions of spherical liquid water droplets. Most derivations follow the general framework of \citep{Beheng01} and \citep{Beheng10}. The goal is to provide a general mathematical framework for bulk-moment microphysics in such a way that the governing equations for bulk moments of the droplet size distribution can be easily related to the detailed properties of collision and breakup kernels. \hl{Clare's comment: I think we should be careful/consistent with terminology. Here you say ``breakup kernel'', but in Eq. 1.4-1.5 it seems like the kernel is just for the collisions and the coalescence/breakup is described by the interaction functions $\chi$ and efficiencies $E$. So there actually is no ``breakup kernel'' in this terminology.}

\section{Governing Equations}
We follow \citep{} in that we write the flux-form evolution equation for the droplet size-distribution function $f(m)$ of liquid water droplets, where $m$ is the droplet mass, as 
\begin{equation}
    \frac{\partial f}{\partial t} +\nabla \cdot (\mathbf{u} f) = \frac{\partial f}{\partial t} \biggr\rvert_{cond} + \frac{\partial f}{\partial t} \biggr\rvert_{sedi} + \frac{\partial f}{\partial t} \biggr\rvert_{coll}, 
    \label{eq:flux_form_f}
\end{equation}
where the dependence of $f(m)$ on $m$ is suppressed in the notation. Here, $\mathbf{u}$ is the barycentric velocity field of the fluid flow (e.g., at the center of mass of the grid box). In writing relation \eqref{eq:flux_form_f} in this form, we implicitly assumed that the size distribution function can be written as a smooth (e.g., $f \in \mathcal{C}^\infty(\mathbb{R})$) function. This approximation is reasonable if there exist enough particles for every value of the mass $m$ in a given grid box \citep[cf.][]{Beheng10}. In what follows, we will use $f$ and $f(m)$ interchangeably to make the notation more compact. The source-sink terms on the right-hand-side of equation~\eqref{eq:flux_form_f} are given by
\begin{equation}
    \frac{\partial f}{\partial t} \biggr\rvert_{cond} = \frac{\partial (\dot{m} f)}{\partial m}
    \label{eq:source_evap}
\end{equation}
for droplet condensation and evaporation. Here, $\dot{m} = \frac{\partial m}{\partial t}$ is the change of mass due to condensation of water vapor onto the droplet and evaporation from the droplet. The source-sink term 
\begin{equation}
    \frac{\partial f}{\partial t} \biggr\rvert_{sedi} = -\frac{\partial(v_sf)}{\partial z}
    \label{eq:source_sedi}
\end{equation}
represents sedimentation processes under the influence of gravity, with $v_s = v_s(m)$ denoting the settling velocity of droplets. $v_s$ is typically determined by the force balance of friction due to drag forces onto the falling droplets and gravity (cf. section \ref{sec:sedimentation}). The last term in equation~\eqref{eq:flux_form_f} can be written as
\begin{equation}
    \frac{\partial f}{\partial t} \biggr\rvert_{coll} = \frac{1}{2} \int_0^\infty \chi(m,m',m'') \mathcal{K}(m', m'')f(m')f(m'') \,\text{d}m' \, \text{d}{m''}
    \label{eq:source_cb}
\end{equation}
and it parameterizes collisional processes for droplet-droplet collisions. We neglect three-droplet collisions as they are less important due to the lower probability of three droplets colliding. Following the convention in \citep{Beheng10}, we call $\chi$ the collisional interaction function and $\mathcal{K}$ the collision kernel. We can write $\chi$ as \hl{Ryan's comment: What is the significance in defining $\mathcal{K}$ for the collision kernel in addition to a function chi (the collisional interaction function). Namely, what is differentiating these two, considering the former descriptor remains unused?}
\begin{subequations}\label{eq:collisional_interaction_function}
\begin{align}
    \chi(m,m',m'') &= \chi_\mathrm{CA}(m,m',m'')  + \chi_\mathrm{BR}(m,m',m'')\\
    \chi_\mathrm{CA}(m,m',m'') &= E_\mathrm{CA}(m',m'')\delta(m-m'-m'')\\
    \nonumber &- \delta(m-m') - \delta(m-m'')\\
    \chi_\mathrm{BR}(m,m',m'') &= E_\mathrm{BR}(m',m'')P(m,m',m'')\\
    1 &= E_\mathrm{CA}(m',m'') + E_\mathrm{BR}(m',m'').
\end{align}
\end{subequations}
The collision kernel $\chi(m,m',m'')$ outputs the number of drops with mass in the interval $[m, m+dm]$ that are formed in a collision of two drops with initial masses $[m', m'+dm]$ and $[m'', m''+dm]$, respectively. The first equation implies that once a collision between two droplets of masses $m'$ and $m''$ has occurred, those droplets either form a new droplet via coalescence (subscript ``CA''), or multiple new droplets are formed via breakup (subscript ``BR''). The form of the second equation with its Dirac delta functions ensures that the collision with subsequent coalescence indeed leads to a new droplet of mass $m' + m''$ while reducing the number of droplets with masses $m'$ and $m''$ appropriately. The coalescence efficiency $E_\mathrm{CA}$ can be viewed as a rate function for the occurrence of this process. The third equation models breakup processes with $E_\mathrm{BR}$ being the breakup efficiency and $P(m, m', m'')$ the fragment size distribution post collision. The fourth equation represents the fact that after a collision has occurred, either a coalescence or a breakup event must follow. %The special case of collision without coalescence or breaking up is implicitly handled by $\chi_\mathrm{BR} $.

Unfortunately for us, the droplet size distribution $f$ is an infinite-dimensional object and must be approximated. This can be achieved by so-called bin schemes, which discretize $f$ in mass-space and evaluate the microphysical processes of the particles in each bin \citep[cf.][]{}. Alternatively, we can write down the governing equations for the moments of the size distribution function $f$ and work with them instead. The moment equations and source-sink terms corresponding to equation~\eqref{eq:flux_form_f} can be obtained by multiplying all terms in \eqref{eq:flux_form_f} by $m^k$ and integration over $m$ from $0$ to $\infty$. This yields a new set of equation that reads
\begin{equation}
    \frac{\partial M_k}{\partial t} +\nabla \cdot (\mathbf{u} M_k) = \frac{\partial M_k}{\partial  t} \biggr\rvert_{cond} + \frac{\partial M_k}{\partial t} \biggr\rvert_{sedi} + \frac{\partial M_k}{\partial t} \biggr\rvert_{coll},
\label{eq:flux_form_mom}
\end{equation}
where $M_k$ now denotes the $k$-th moment of $f$. Since $k \in \mathbb{N}_0$, this formally represents a set of infinitely many equations. The source-sink terms for each moment $M_k$ can be written correspondingly for evaporation and condensation as
\begin{equation}
    \frac{\partial M_k}{\partial t} \biggr\rvert_{cond} = k\int_0^\infty m^{k-1}\dot{m}f(m) \text{dm}, 
\label{eq:source_evap_mom}
\end{equation}
where the expression was obtain by integration by parts with respect to $m$ under the integral. The source-sink term for for droplet sedimentation can be written as
\begin{equation}
    \frac{\partial M_k}{\partial t} \biggr\rvert_{sedi} = -\frac{\partial}{\partial z}\int_0^\infty m^k v_s(m) f(m) \text{d}m,
\label{eq:source_sedi_mom}
\end{equation}
and finally, the source-sink term for two-particle collisional processes can be written as
\begin{equation}
    \frac{\partial M_k}{\partial t} \biggr\rvert_{coll} = \frac{1}{2} \int_0^\infty m^k \chi(m,m',m'') \mathcal{K}(m', m'')f(m')f(m'') \, \text{d}m\, \text{d}m' \,\text{d}{m''}.
\label{eq:source_cb_mom}
\end{equation}
In what follows, we will make assumptions and simplifications that reduce the complexity of these source-sink terms sufficiently for modeling purposes within GCM and LES models. We seek an approximation scheme that is mathematically and physically consistent for different levels of approximation complexity.

\section{Condensation \& Evaporation of Droplets}
\label{sec:condensation_evaporation}
A cloud particle can grow by condensation of water vapor onto its surface, resulting in an increase of the particle's mass. Conversely, it can lose mass due to evaporation of water away from its surface into the surrounding air.
To derive an expression for the rate of change of a droplet's mass due to the net effect of condensation and evaporation of water molecules, we consider a water droplet of radius $r_{d}$ in moist air with water vapor density $\rho_v$, and we assume that the air is moving with velocity $\mathbf{u}$ relative to the droplet. The total flux density of water vapor toward the drop surface, $\mathbf{j_v}$, is given by: 
\begin{equation}
    \mathbf{j_v} = -D_v \nabla \rho_v + \rho_v \mathbf{u},
\label{eq:total_vapor_flux_density}
\end{equation}
where $D_v$ is the diffusivity of water vapor in air. The first term on the right-hand side of equation (\ref{eq:total_vapor_flux_density}) is the contribution due to molecular diffusion, which is assumed to follow Fick's first law (the flux goes from regions of high concentration to regions of low concentration, with a magnitude that is proportional to the concentration gradient). The second term is the convective flux density, which represents the flux density due to the motion of the drop. 

The total water vapor mass is conserved, and so the time rate of change of water vapor mass in a Eulerian control volume is equal to the total water vapor mass flux leaving the surface of the volume:
\begin{equation}
    \int_V \frac{\partial \rho_v}{\partial t} dV = - \int_{\mathbf{S}} \mathbf{j_v} \cdot ~d\mathbf{S} ~=~ - \int_V \nabla \cdot \mathbf{j_v} ~dV,
    \label{eq:water_vapor_mass_conservation}
\end{equation}
where the second equality follows from the divergence theorem. Substituting equation (\ref{eq:total_vapor_flux_density}) into equation (\ref{eq:water_vapor_mass_conservation}) and assuming incompressible flow ($\nabla \cdot \mathbf{u} = 0$) results in:
\begin{equation}
    \frac{\partial \rho_v}{\partial t} = D_v \nabla^2 \rho_v - \mathbf{u} \cdot \nabla \rho_v,
    \label{eq:water_vapor_density_evolution}
\end{equation}
where we have dropped the integrals since equation (\ref{eq:water_vapor_mass_conservation}) holds for arbitrary volumes.
This equation describes how the water vapor density in the vicinity of a moving (falling) water droplet changes with time. Assuming that the droplet is stationary with respect to the flow ($\mathbf{u} = \mathbf{0}$) and that $\rho_v$ is in steady-state ($\partial \rho_v / \partial t = 0$), equation (\ref{eq:water_vapor_density_evolution}) reduces to Laplace's equation,
\begin{equation}
    \nabla^2 \rho_v = 0.
    \label{eq:laplace}
\end{equation}
As long as the time scale of the droplet's diffusional growth/reduction is faster than that of (major) fluctuations in the environmental water vapor density (and in most atmospheric flow settings it is), the steady-state assumption is a good approximation. Transformation to spherical coordinates gives the following equation for the radial component of equation~(\ref{eq:laplace}):
\begin{equation}
    \frac{1}{r^2}\frac{\partial}{\partial r} \left(r^2 \frac{\partial \rho_v}{\partial r} \right)
    \label{eq:laplace_radial}
\end{equation}
At the droplet surface, $\rho_v = \rho_{v, r_d}$ (which is usually assumed to be the saturated water vapor density), and for $r \rightarrow \infty$, $\rho_v = \rho_{v, \infty}$ (a constant water vapor density far away from the droplet). With these boundary conditions, the solution to equation~(\ref{eq:laplace_radial}) becomes:
\begin{equation}
    \rho_v = \rho_{v, \infty} + (\rho_{v, r_d} - \rho_{v, \infty}) \frac{r_{d}}{r} = 0, 
    \label{eq:laplace_radial_solution}
\end{equation}
Our final goal is an expression for $dm/dt$, the time rate of change of a drop's mass due to condensation and evaporation of water vapor. This rate is equal to the integral of the water vapor flux density, $D_v \nabla \rho_v$ [where $\nabla \rho_v$ is obtained from equation~(\ref{eq:laplace_radial_solution})], over the surface of the drop. This yields
\begin{equation}
    \frac{dm}{dt} = 4 \pi D_v r_d (\rho_{v, \infty} - \rho_{v, r_d}) 
    \label{eq:condens_dm_dt}
\end{equation}
When $\rho_{v, \infty}$ exceeds $\rho_{v, r_d}$, the drop will gain mass, and conversely it will lose mass. In this derivation we have made the assumption that the diffusivity $D_v$ is the same in free air as it is just above the droplet surface, where water vapor may behave more like an ensemble of molecules than a continuum. We have also neglected latent heat effects, i.e., we have assumed that the temperature at the surface of the droplet is the same as that of the ambient air. In reality, the temperature at the surface of a growing droplet is warmer due to the latent heat release associated with condensation of water vapor. This increases the saturation vapor pressure above the droplet surface, decreasing the difference between $\rho_{v, \infty}$ and $\rho_{v, r_d}=\rho_{sat, r_d}$ in equation~({\ref{eq:condens_dm_dt}}), thereby slowing the droplet's condensational growth. The latent heat effect is a significant factor in diffusional droplet growth -- in the following, we thus expand equation (\ref{eq:condens_dm_dt}) to include this process.

A continued diffusion of water vapor molecules onto the droplet surface requires diffusion of enthalpy (heat) in the opposite direction. In analogy to equation (\ref{eq:condens_dm_dt}), the steady-state equation for diffusion of heat ($q$) toward the droplet is given by
\begin{equation}
\frac{dq}{dt} = 4 \pi K r_d (T_{\infty} - T_{r_d}),
\label{eq:dq_dt}
\end{equation}
where $K$ is the thermal conductivity coefficient, $T_\infty$ is the environmental temperature, and $T_{r_d}$ is the temperature at the drop's surface. Note that in most drop diffusion growth problems, $T_\infty$ is given but $T_{r_d}$ needs to be determined.

The difference between the rate of latent heat production at the drop's surface and the rate at which heat is transported away from the drop determines the temperature change at the drop's surface:
\begin{equation}
\frac{4 \pi}{3} r_d^3 \rho_l c_{pw} \frac{dT_{r_d}}{dt} = L_v \frac{dm}{dt} - \frac{dq}{dt},
\label{eq:dT_dt_at_drop_surface}
\end{equation}
where $c_{pw}$ is the is the specific heat of water at constant pressure and $L_v$ is the latent heat of vaporization. Assuming a steady-state-diffusion process ($\frac{dT_{r_d}}{dt} = 0$), temperature and density fields are then related according to
\begin{equation}
L_v D_v  (\rho_{v, \infty} - \rho_{v, r_d}) = K (T_{\infty} - T_{r_d})
\label{eq:steady_state_diffusion_condition}
\end{equation}
Recalling that $\rho_{v, r_d}=\rho_{sat, r_d}$, which -- for a given drop size, solute amount and type -- is only a function of $T_{r_d}$, we note that equation (\ref{eq:steady_state_diffusion_condition}) is an implicit
equation of $T_{r_d}$, or equivalently of $\rho_{v, r_d}$. Together with equation (\ref{eq:condens_dm_dt}), it forms a coupled system with no known analytical solutions. One way to proceed is to do a first-order Taylor expansion of $\rho_{v, r_d}$ about the ambient temperature $T_{\infty}$ (apologies for the sloppy use of the equality sign):
\begin{equation}
    \rho_{v, r_d}(T_{r_d}) = \rho_{v, r_d}(T_{\infty}) + \frac{d}{dT}\biggr\rvert_{T_{\infty}} (T_{r_d} - T)
\end{equation}
Using equation (\ref{eq:dq_dt}) and equation (\ref{eq:steady_state_diffusion_condition}), $(T_{r_d} - T)$ can be expressed in terms of the condensation rate, and we obtain 
\begin{equation}
    \rho_{v, r_d}(T_{r_d}) = \rho_{v, r_d}(T_{\infty}) + \frac{d\rho_{sat, r_d}}{dT}\biggr\rvert_{T_{\infty}} \left(\frac{L_v}{4\pi r_d K}\frac{dm}{dT}\right)
\end{equation}
Substituting this expression into equation (\ref{eq:condens_dm_dt}) and solving for $\frac{dm}{dt}$, we get
\begin{equation}
    \frac{dm}{dt} = 4 \pi r_d D_v \left(\frac{\rho_{v, \infty} - \rho_{sat, r_d}(T_\infty)}{1 + \frac{D_v L_v}{K} + \frac{d\rho_{sat, r_d}}{dT}\rvert_{T_{\infty}}}\right)
\end{equation}

This equation describes the diffusional growth (or shrinkage) of a droplet of radius $r_d$ in a vapor field of density $\rho_{v, \infty}$ and temperature $T_\infty$. The final step is to derive an expression for $\frac{d\rho_{sat, r_d}}{dT}\biggr\rvert_{T_{\infty}}$. Assuming that the drop is large enough for the curvature and solute effects to be negligible, we can make use of the Clausius-Clapeyron relation, $\frac{d ln(e_{sat})}{dT} = \frac{L_v}{R_v T^2}$ (where $e_{sat}$ is the saturation water vapor pressure and $R_v$ is the specific gas constant of water vapor):
\begin{equation}
    \frac{d\rho_{sat, r_d}}{dT}\biggr\rvert_{T_{\infty}} = \frac{1}{R_v T_\infty} \frac{d e_*(T_\infty}{dT} - \frac{e_*(_\infty)}{R_v T_\infty^2} = \frac{e_*}{R_v T_\infty^2} \left(\frac{L_v}{R_v T_\infty} - 1 \right),
\label{eq:claus_clap}
\end{equation}
where $e_*$ is the ``base value'' of $e_{sat}$, i.e., the value for pure water in the absence of surface tension effects. Finally, we can combine the last two equations into an expression for diffusional growth that includes latent heat effects:
\begin{equation}
    \frac{dm}{dt} = 4 \pi r_d \left(\frac{\mathcal{S} - 1}{F_D + F_K}\right), 
\end{equation}
with
\begin{equation}
    F_K(T) = \frac{L_v \rho_l}{K T} \left(\frac{L_v}{R_vT} - 1\right) ~\textrm{and} ~~ F_D(T) = \frac{R_v T \rho_l}{D_v e_*(T)}
\end{equation}
Using equation (\ref{eq:condens_dm_dt}), the moments' form of the contribution of condensation and evaporation becomes:
\begin{equation}
\begin{aligned}
    \frac{\partial M_k}{\partial t} \biggr\rvert_{cond} &= k\int_0^\infty m^{k-1}\dot{m}f(m) \text{dm} \\
                                                        &= k\int_0^\infty m^{k-1}4 \pi r_d \left(\frac{\mathcal{S} - 1}{F_D + F_K}\right)f(m) \text{dm}
\label{eq:condensation_moments}
\end{aligned}
\end{equation}
\noindent
\colorbox{pink}{\parbox{12cm}{Anna's comment: I don't think that neglecting latent heat effects is a good approximation. 
You could look at the derivation for eq 5.106 in the book by Jerry Straka Cloud and Precipitation 
Microphysics Principles and Parameterizations. Or in chapter 16 in Jacobson, M.: Fundamentals of Atmospheric Modelling. The general idea is to write similar equations for the diffusion of heat,
use Clausius-Clapeyron relation, assume that the temperature at the drop surface is close to the ambient temperature and water vapor density at the drop surface is close to saturation.
After all of that you will end up with an equation for drop radius like eq (11) in Microphysics
docs for CLIMA}}
\colorbox{orange}{\parbox{12cm}{Melanie's comment: Included latent heat effects. Other processes that may/should be included at some point: curvature and solute effects, molecular kinetic effects (for small droplets, continuum description may not be a good approximation), ventilation effects (when a drop is large enough to fall trough the vapor field, the equilibrium assumption for the water vapor field in the drop's vicinity may not be well justified).}}


\section{Sedimentation of Droplets}
\label{sec:sedimentation}
Since water is denser than air, cloud droplets are always falling relative to the air. When a cloud appears to be floating in the air without any vertical motion, it is because the updrafts in the cloud are strong enough to balance the gravitational pull acting on the cloud droplets.
In general, a droplet falling through air experiences an upward acceleration due to its buoyancy ($\mathbf{F_B}$) and the hydrodynamic drag force ($\mathbf{F_D})$, and a downward acceleration due to gravity ($\mathbf{F_G}$) -- the equation of motion for the droplet is thus:
\begin{equation}
    m\frac{d\mathbf{u}}{dt} = \mathbf{F_B} + \mathbf{F_D} + \mathbf{F_G}
\label{eq:forces_falling_droplet}
\end{equation}
The magnitudes (z-components) of these forces are given by:
\begin{equation}
\begin{aligned}
    F_B &= \frac{4}{3} \pi r^3 \rho_a g \\
    F_D &= C_D \frac{1}{2} \rho_a v^2 A \\
    F_G &= -\frac{4}{3} \pi r^3 \rho_l g
\label{eq:forces_definitions}
\end{aligned}
\end{equation}
where $r$ is the droplet radius, $\rho_a$ is the air density, $\rho_l$ is the density of water, $C_D$ is the drag coefficient, $v$ is the droplet's vertical velocity, and $A = \pi r^2$ is the droplet's cross-sectional area.
When the right-hand side of equation~(\ref{eq:forces_falling_droplet}) vanishes, we have that $\frac{d\mathbf{u}}{dt} = 0$, i.e., the droplet falls at constant velocity relative to the surrounding air. This steady-state (or terminal) settling  velocity, denoted $v_s$, is obtained by setting the left-hand side of equation~(\ref{eq:forces_falling_droplet}) to zero and solving for $v$. This yields
\begin{equation}
    v_s = \sqrt{\frac{8 g r}{3 \, C_{D}} \left( \frac{\rho_{w} - \rho_a}{\rho_a} \right)}
\label{eq:settling_velocity}
\end{equation}
Note that the drag coefficient depends itself on the velocity, so in the general case, equation~(\ref{eq:settling_velocity}) represents an implicit equation for $v_s$. There are different approximations and empirical formulas for determining the settling velocities of droplets of various sizes falling at different pressure levels. For example, in the Stokes regime (Reynolds numbers $\ll 1$), inertial forces can be ignored, and the drag is given by $F_D = 6 \pi \eta r v_s$ (where $\eta$ is the dynamic viscosity of air). In this regime, which is a good approximation for falling cloud droplet of less than a few tens of micrometers in radius, the drag force is thus  proportional to the speed (rather than the square of the speed), and $v_s$ is given explicitly by:
\begin{equation}
    v_{s, Stokes} = \frac{2 r^2 g (\rho_l - \rho_a)}{9 \eta}
\label{eq:settling_velocity_stokes}
\end{equation}
For larger drops, there are no simple drag expressions, and drag coefficients have to be modelled using machine learning and/or experimental results.

Another point worth mentioning is that we assume cloud and raindrops to always fall at their terminal velocities, whereas in reality it takes some time for each drop to reach its terminal velocity. Since the terminal velocity represents an upper bound to a drop's fall speed, this assumption will lead to an overestimation of the average droplet fall speed.

With the expression for $v_s$ derived in this section and writing it as a function of mass instead of radius (for a spherical water droplet with radius $r$, $m = \frac{4 \pi}{3} r^3 \rho_l$), the moments' form of the contribution of sedimentation under the influence of gravity becomes:

\begin{equation}
\begin{aligned}
    \frac{\partial M_k}{\partial t} \biggr\rvert_{sedi} &= -\frac{\partial}{\partial z}\int_0^\infty m^k v_s(m) f(m) \text{d}m \\
                                                        &= -\frac{\partial}{\partial z}\int_0^\infty m^k \sqrt{\frac{8 g \left(\frac{3 m}{4 \pi \rho_l}\right)^{\frac{1}{3}}}{3 C_{D}} \left( \frac{\rho_{w} - \rho_a}{\rho_a} \right)} f(m) \text{d}m
\end{aligned}
\label{eq:sedimentation_moments}
\end{equation}

\subsection{Beard 1977 terminal velocity}

\citet{Beard1977} provides formulae for terminal velocities of water drops calculated as a numerical fit  to measurements. Water drops are divided into two categories based on their diameter $ d \in (1 \mu m,  40 \mu m)$ and $d \in (40 \mu m, 6mm)$. The terminal velocity is first computed at the sea level (for $p = 1013 \; hPa$, $T = 293.15 \; K$ and $\rho = 1.194 \; kg/m3$ and then adjusted for the conditions aloft using a correction factor that depends on $T$, $p$ and $d$:
\begin{equation}
    v_{B77} = v_{sea} * f_{corr},
\end{equation}
where $v_{sea}$ is the terminal velocity at the sea level and $f_{corr}$ is the correction factor.
The velocity at the sea level is computed as:
\begin{equation}
\begin{aligned}
    v_{sea} &= exp(y),\\
    y &= \sum_{i=0}^{m} c_i x^i, \\
    x &= ln(d),
\end{aligned}
\end{equation}
where $d$ is the drop diameter in $\mu m$. The parameters for the two size regimes are: (10.5035,  1.08750, -0.133245, -0.00659969) and (6.5639, -1.0391, -1.4001, -0.82736, -0.34277, -0.083072, -0.010583, -0.00054208). The correction factor for the smaller drops is computed as:
\begin{equation}
    f_{corr} = \frac{\eta_0}{\eta} \; \frac{d + 2.51 l}{d + 2.51 l_0},
\end{equation}
where $\eta$ is the dynamic viscosity,
$\eta_0 = 1.818 \;\; 10^{-5} \; Pa \; s$,
$l$ is the mean free path, and
$l_0 = 6.62 \; 10^{-8} \; m$.
The mean free path is computed as:
\begin{equation}
l = l_0 \frac{\eta}{\eta_0}  \sqrt{\frac{p_{0} \rho_{0}}{\rho_a p}}
\end{equation}
where $p_0 = 1013 \; hPa$ and $\rho_0 = 1.20 \; kg/m3$. {\color{red} TODO - the density is different than the sea level density from table...} The correction factor for the big drop category is computed as:
\begin{equation}
f_{corr} = 1 + 1.104 \; \eps_s + ( 1.058 \; \eps_c - 1.104 \; \eps_s) \; (5.52 + ln(d))/5.01),
\end{equation}
where
\begin{equation}
  \eps_s = \frac{\eta_0}{\eta} - 1,
\end{equation}
\begin{equation}
  \eps_c = \sqrt{\frac{\rho_0}{\rho_a}} - 1.
\end{equation}
{\color{red} TODO - figure out what units and what log are used for ln(d)}



\section{Collisions}
In this section we describe approximations for collisional processes when breakups of particles after collision can be neglected. This is an assumption that does not always hold and it will need to be revisited. Generally speaking, the mathematics presented in this section will apply to the more general case of collisions with breakups as well, but will be more tedious. If we assume that droplets cannot break into smaller droplets, equation \eqref{eq:source_cb_mom} simplifies to a double integral and is easy to handle with a polynomial approximation to the collision kernel and coalescence efficiency. For simplicity, we denote the product of the collision kernel $\mathcal{K}$ and the coalescence efficiency $E_\mathrm{CA}$ as $\mathcal{C}$. We can then write
\begin{equation}
    \frac{\partial M_k}{\partial t} \biggr\rvert_{coll} = \frac{1}{2}\int_0^\infty \left((m+m')^k - m^k - {m'}^k\right) \mathcal{C}(m, m')f(m)f(m') \, \text{d}m\, \text{d}m',
\label{eq:source_co_mom}
\end{equation}
where $\mathcal{C}(m, m') = E_{CA}(m, m') \mathcal{K}(m, m')$ is the product of the collision kernel with the coalescence efficiency. In order to avoid confusion with terminology, we call $\mathcal{C}$ the collision-coalescence kernel. We seek an approximation to equation~\eqref{eq:source_co_mom} that lends itself to efficient implementations.

\subsection{Polynomial Kernel Approximations \& Closure}
\label{subsec:poly_kernel_approx}
In order to simplify the integral expression in equation \eqref{eq:source_co_mom} in terms of the moments $M_k$ of $f(m)$, we can approximate the collision-coalescence kernel via a $2r$-order polynomial through an expansion in monomials. In order to simplify relation~\eqref{eq:source_co_mom}, we make the approximation
\begin{equation}
    \mathcal{C}(m,m') \approx \sum_{a,b=0}^{r} c_{ab} m^a{m'}^b, \quad \text{with} \quad r \geq 0
\label{eq:dist_approx}
\end{equation} 

 The matrix defined by the $c_{ab}$ can be seen as a lookup table for the collision-coalescence kernel $\mathcal{C}$. It must obey $c_{ab} = c_{ba}$ because $\mathcal{C}$ is symmetric in $m$ and $m'$. 

Plugging the approximation (\ref{eq:dist_approx}) into equation (\ref{eq:source_co_mom}) and applying the binomial formula, $(m+m')^k = \sum_{i=0}^k {k \choose i} m^i m'^{k-i}$, we get:

\begin{equation}
\begin{aligned}
    \frac{\partial M_k}{\partial t} \biggr\rvert_{coll} &\approx \frac{1}{2}\int_0^\infty \left((m+m')^k - m^k - {m'}^k\right) \left(\sum_{a,b=0}^{r} c_{ab} m^a{m'}^b\right)f(m)f(m') \, \text{d}m\, \text{d}m' \\
    &= \frac{1}{2}\int_0^\infty \left(\sum_{i=0}^k {k \choose i} m^i m'^{k-i} - m^k - {m'}^k\right) \left(\sum_{a,b=0}^{r} c_{ab} m^a{m'}^b\right)f(m)f(m') \, \text{d}m\, \text{d}m' \\
    &= \frac{1}{2}\int_0^\infty \left(\sum_{a,b=0}^{r}\sum_{c=0}^{k} c_{ab} {k \choose c} m^{a+c}m'^{b+k-c} \right) f(m) f(m')\, \text{d}m\, \text{d}m' \\ &~~~~~~~~~~~- \frac{1}{2}\int_0^\infty \sum_{a,b=0}^{r} m^{a+k} m'^b\, \text{d}m\, \text{d}m' - \frac{1}{2}\int_0^\infty \sum_{a,b=0}^{r} m^{a} m'^{b+k}\, \text{d}m\, \text{d}m'
\label{eq:derivation_source_co_mom_approx}
\end{aligned}
\end{equation}

\noindent We can simplify further by using that $\int_0^\infty m^j f(m)\, \text{d}m \equiv M_j$:
\begin{equation}
    \frac{\partial M_k}{\partial t} \biggr\rvert_{coll} \approx
    \frac{1}{2}\sum_{a,b=0}^{r}\sum_{c=0}^{k} c_{ab} {k \choose c} M_{a+c}M_{b+k-c} - \frac{1}{2} \sum_{a,b=0}^{r} M_{a+k} M_b - \frac{1}{2}\sum_{a,b=0}^{r} M_{a} M_{b+k}
\label{eq:derivation_source_co_mom_approx_contd}
\end{equation}

\noindent And since $\sum_{a,b=0}^{r} M_{a+k} M_b = \sum_{a,b=0}^{r} M_{a} M_{b+k}$, we arrive at the expression 

\begin{equation}
\frac{\partial M_k}{\partial t} \biggr\rvert_{coll} \approx
    \frac{1}{2} \sum_{a,b=0}^{r} \sum_{c=0}^k c_{ab} {k \choose c} M_{a+c} M_{b+k-c} - \sum_{a,b=0}^{r} c_{ab} M_{a+k} M_b
\label{eq:source_co_mom_approx}
\end{equation}
for the collisional source term, i.e., the time rate of change of the  moment $M_k$ due to collision-coalescence processes (note that the factor $0.5$ avoids double counting). For the special cases of $k=0$ or $k=1$ we have simplified relations
\begin{equation}
    \frac{\partial M_k}{\partial t} \biggr\rvert_{coll} \approx
    \begin{cases}
    -\frac{1}{2} \sum_{a,b=0}^{r} c_{ab} M_a M_b & \quad k=0\\
    0 & \quad k = 1.
    \end{cases}
\end{equation}
Here, the second relationship simply reflects the fact that collision-coalescence processes conserve mass. These equations are obviously not new and are derived in numerous sources \citep[cf.][]{Pruppacher1978, Beheng2010}. They relate the source-sink term for the moment $M_k$ to the moments $M_0, \dots, M_{r+k}$, which usually means that the time evolution of $M_k$ depends on higher moments $M_q$ with $q>k$. This represents a classic closure problem for the moments of $f$ and needs to be addressed by any moment-based microphysics scheme with nontrivial $\mathcal{C}$.

The fact that we are in a situation with an infinite hierarchy for the moments $M_k$ with closure problem is related to a) the fact that collisional interactions are nonlinear in $f$, b) the nontrivial dependence of most kernels $\mathcal{C}$ on $m$ and $m'$, and c) the infinite-dimensional nature of $f$. We have addressed b) by approximating the kernel with a polynomial. Now we need to reduce the allowed complexity (e.g., the allowed shape) of the size distribution function $f$ in order to close the infinite hierarchy. We choose to approximate $f$ via
\begin{equation}
    f(m) \approx \hat{f}(m, \theta), \quad \text{with} \quad \theta \in \mathbb{R}^{s}.
\end{equation}
If we require that all moments of $f$ and $\hat{f}$ are finite, we can formally write the following approximation for the moments of $f$
\begin{equation}
    M_k \approx g_k(\theta) = \int_0^\infty m^k\hat{f}(m, \theta) \, \text{d}m.
\end{equation}
One can recognize that the approximation $\hat{f}$ of $f$ is constrained by the first $s$ moments $M_0, \dots, M_{s-1}$ because $\theta \in \mathbb{R}^s$. For simplicity, let us write the first $s$ moments of $f$ in vector form as $\mathbf{M}_p$ and the next $r$ moments of $f$ as $\mathbf{M}_d$, where the p stands for ``prognostic'' and the d stands for ``diagnostic''. This separation highlights the fact that we should only have to evolve in time the first $s$ moments of $f$ in order to constrain the approximation $\hat{f}$ appropriately. Then, according to equation~\eqref{eq:source_co_mom_approx} the highest moment occurring in the time evolution equation for the $s-1-th$ moment $M_{s-1}$ is $M_{r+s-1}$. More formally, we have relations
\begin{subequations}\label{eq:closure}
\begin{align}
    \mathbf{M}_p &= \mathbf{G}_p(\theta), \quad \text{with} \quad \mathbf{M}_p \in \mathbb{R}^s \\
    \mathbf{M}_d &= \mathbf{G}_d(\theta), \quad \text{with} \quad \mathbf{M}_d \in \mathbb{R}^r,
\end{align}
\end{subequations}
where the $\mathbf{G}_{(.)}$ (with $(.)$ either p or d) are vector-valued functions of $\theta$. We can immediately recognize that for $r = 0$ we do not need to care about $\mathbf{M}_d$. This is because the lower moments do not couple to higher moments in equation~\eqref{eq:source_co_mom_approx}. At the level of the time evolution, a closure is realized by time stepping the first $s$ moments and by inverting the first relation in \eqref{eq:closure} to get $\theta$ at each time step and then evaluating the second relation in \eqref{eq:closure} to get all moments necessary to evaluate the the source-sink due to collision-coalesence (e.g., \eqref{eq:source_co_mom_approx}) for the first $s$ moments.

From a user perspective, one only needs to provide the lookup table defined by the $c_{ab}$, and the analytic moment relations \eqref{eq:closure} for a chosen approximating distribution function $\hat{f}$. A choice has to be made as to how to invert the first relation in \eqref{eq:closure} and this is where a price needs to be paid. This is also the point where the presented approach deviates from \cite{Beheng01} and others. Options are to a) numerically determine the value of $\theta$ given $\mathbf{M}_p$ at every time step via an off-the-shelf optimization routine. This costs extra compute cycles for the optimization. Alternative b) would be to time step $\theta$ directly and update the values at every time step. This is likely computationally cheaper. We will outline alternative b) in a section below.

\subsection{Piecewise Polynomial Kernel approximations}
For piecewise polynomial kernels we can make a similar approximation as for polynomial kernels at the cost of introducing moments for the different subcomponents of the distribution function $f$ implied by the piecewise approximation. \hl{Toby's comment: May or may not need this.}

\subsection{Closure Parameter Optimization}
As described in section \ref{subsec:poly_kernel_approx}, the rate of change of a predicted moment
generally depends on moments of higher orders, which presents a closure problem conceptually
similar to the problem of subgrid-scale turbulence closure. 
In bulk microphysics schemes such as the one introduced here, the closure is usually achieved by assuming an
analytic functional form for the droplet size/mass distribution. The analytic expression for the droplet distribution is used to compute the higher-order diagnostic moments $\mathbf{M}_d$ needed to evaluate the time rate of change of the prognostic moments $\mathbf{M}_p$. However, computing the diagnostic moments requires to know the distribution parameters, $\mathbf{\theta}$, which are determined by inverting (\ref{eq:closure}a):

\begin{equation}\label{eq:Mp_to_param}
\mathbf{\theta} = \mathbf{G}_p^{-1}(\mathbf{{M}_p}), ~ \mathbf{M}_p \in \mathbb{R}^s, 
\end{equation}

\noindent where $\mathbf{G}_p^{-1}$ is the inverse of $\mathbf{G}_p$. Except for simple distributions, $\mathbf{G}_p^{-1}$ does not have a closed-form representation -- thus, in the general case, the parameters $\mathbf{\theta}$ are determined by solving an optimization
problem. Alternatively, the function $\mathbf{G}_p^{-1}$ can be approximated by a machine learning model that has been trained to predict the parameters given the prognostic moments as input.

How best to approach the problem of mapping moments to parameters depends both on the type of droplet distribution and on the relative importance of accuracy, speed, ease of implementation, etc.

The following approaches are currently implemented:
\begin{itemize}
    \item \textbf{Analytical solutions}: For gamma distributions and exponential distributions, $\mathbf{G}_p^{-1}$ is available in closed form and provides a straightforward way to compute the underlying distribution parameters from the prognostic moments.
    \item \textbf{Optimization}: Optimization solvers can be used to find the parameters $\widehat{\mathbf{\theta}}^*$ that minimize the distance between the given prognostic moments $\mathbf{M}_p$ and the prognostic moments computed from the estimated parameters $\widehat{\mathbf{\theta}}$, $\widehat{\mathbf{M}}_p$:
\begin{equation}\label{eq:Mp_to_param}
\begin{aligned}
\widehat{\mathbf{\theta}}^* &= \operatorname*{argmin}_{\widehat{\mathbf{\theta}}} \| \mathbf{M}_p - \mathbf{\widehat{M}}_p\| \\
&= \operatorname*{argmin}_{\widehat{\mathbf{\theta}}} \| \mathbf{M}_p - \mathbf{G}_p(\widehat{\mathbf{\theta}})\| \\
&= \operatorname*{argmin}_{\widehat{\mathbf{\theta}}} g(\mathbf{\widehat{\theta}}),
\end{aligned}
\end{equation}
where $g(\mathbf{\widehat{\theta}}) := \| \mathbf{M}_p - \mathbf{G}_p(\widehat{\mathbf{\theta}})\|$, and $\| \cdot \|$ denotes the L2 norm. One may attempt to treat this as an unconstrained problem and use any appropriate global optimization solver to minimize the objective function $g$. This basic approach can be customized to  specific distribution types. For example, for mixtures of gamma distributions and mixtures of exponential distributions, we implemented a nonlinear least squares formulation subject to constraints on the unknowns (e.g., that certain parameters have to be positive), which has shown to improve both accuracy and speed of the optimization procedure.
\end{itemize}

Customized approaches (analytical solutions or optimization procedures) for other distribution types can easily be added, and as mentioned above, a third option would be to get the mapping from moments to parameters from a call to a pre-trained machine learning model. Whether there are distributions where machine learning provides any advantage over the other approaches described here remains to be seen.

\subsection{Closure Parameter Adjustment}
Time stepping $\theta$.

% As a result, the time derivative of $\mathbf{M}_p$ is given by
% \begin{equation}
%     \frac{\partial \mathbf{M}_k}{\partial t} = \mathrm{D}\mathbf{G}(\theta) \cdot \frac{\partial \theta}{\partial t}, \quad \text{with} \quad \mathrm{D}\mathbf{G}(\theta) \in \mathbb{R}^{s \times s},
% \end{equation}
% where $\mathrmD\mathbf{G}(\theta)$ denotes the Jacobian matrix of $\mathbf{G}$. In general this Jacobian should have full rank because $\hat{f}$ should be uniquely constrainable via the first $s$ moments. This assumption is broken, for example, when we deal with general mixture distributions for $\hat{f}$.

\hl{Toby's comment: Will need to split this section into two when sedimentation and condensation has been addressed.}

\section{Parameterizing Collisions using Normalizing Flows}

An alternative way of parameterizing the collision process is with the help of invertible transforms of a base mass distribution. Consider a distribution $n_0(m)$ (e.g., an exponential or gamma distribution). If we have an invertible map $f_\theta:\mathbb{R} \rightarrow \mathbb{R}$, that is parameterized by a parameter vector $\theta$, then we can perform a change of variables that defines a new distribution
\begin{equation}
    \label{eq:normalizing_flow}
    n_\theta(m) = \Big|\frac{\partial f_\theta(m)}{\partial m}\Big| n_0(f_\theta(m)).
\end{equation}
By picking a friendly base distribution $n_0$ it becomes very easy to generate samples from $n_\theta$. One simply samples from $n_0$ and applies the back transformation $f^{-1}$ to generate samples from $n$.

Given that the general collision process is governed by an equation of the form
\begin{equation}
    \label{eq:gen_coll_process}
    \frac{\partial n_\theta(m)}{\partial t} = \int \int K(m, m',m'')n_\theta(m')n_\theta(m'')\, \mathrm{d}m'\mathrm{d}m'',
\end{equation}
we can design a potentially efficient way of solving \eqref{eq:gen_coll_process} using the normalizing flow definition given by \eqref{eq:normalizing_flow}.

\subsection{Kernel Library}
TBD

\subsubsection{Gravitational Collision Kernel}
TBD

\subsubsection{Turbulent Collision Kernels}
TBD

\chapter{Ice-Phase Microphysics Scheme}
TBD

\chapter{Aerosol Microphysics Scheme}
TBD

%\end{subsection}

%-------Bibliography
\bibliographystyle{agufull08}
\bibliography{Giraldo_refs,CLIMA-refs}

\end{document}