\documentclass[9pt]{article}
\usepackage{amsmath,amsthm,amsfonts,amssymb,epsfig}
\usepackage{color}
\usepackage{subfig}
\usepackage{float}
\usepackage{authblk}
\usepackage{wrapfig}
\usepackage{multicol}
\bibliographystyle{plain}

\textheight 24cm
\textwidth 15cm
\topmargin -1.5cm
\oddsidemargin -0.1cm
\evensidemargin 5cm

\newcommand{\reals}{\mathbb{R}}
\newcommand{\naturals}{\mathbb{N}}
\newcommand{\complex}{\mathbb{C}}
\newcommand{\integers}{\mathbb{Z}}
\newcommand{\banach}{\mathbb{B}}
\newcommand{\exponent}{\operatorname{e}}
\newcommand{\diag}{\operatorname{diag}}
\newcommand{\interior}{\operatorname{int}}
\newcommand{\deter}{\operatorname{det}}



%\theoremstyle{plain}
%\newtheorem{defi}{Definition}
%\newtheorem{prop}{Proposition}
\newtheorem{stel}{Theorem}
\newtheorem{gevolg}{Corollary}
\newtheorem{lemma}{Lemma}
\theoremstyle{remark}
\newtheorem{opm}{Remark}
\newtheorem{defi}{Definition}

\begin{document}

\title{\bf Viscous b.c. in CLIMA}

\author[]{}
%\affil[1]{}
\date{}

\maketitle

By physics, the following b.c. should be applied for viscous flows:

\begin{itemize}
\item NO-Slip condition:
    \begin{equation}
    \rho^+ = \rho^- \quad\quad {\rm Free}
    \end{equation}
    \begin{equation}
    {\bf u}^+ = 0 \quad\quad {\rm no slip}
    \end{equation}
    \begin{equation}
    E^+ = E^- \quad\quad {\rm Free}
    \end{equation}
    
    This may be achieved in CLIMA by imposing the following conditions:
    
    \begin{equation}
    QP[_\rho]= QM[_\rho]
    \end{equation}
    \begin{equation}
    QP[{\rm _U}] = 0
    \end{equation}
    \begin{equation}
    QP[_E] = QM[_E] 
    \end{equation}
    
{\color{red}This option {\bf blows up} with big error at the boundaries.}

\item FREE-slip condition:

    \begin{equation}
    \rho^+ = \rho^- \quad\quad {\rm Free}
    \end{equation}
    \begin{equation}
    {\bf u}\cdot{\bf n} = 0 \quad\quad {\rm no-penetration,\,no\,shear\,stress}
    \end{equation}
    \begin{equation}
    E^+ = E^- \quad\quad {\rm Free}
    \end{equation}
    
    This may be achieved in CLIMA in either one of two ways:
    \begin{itemize}
	\item[a]
                \begin{equation}
                (QP[_\rho]= QM[_\rho])
                \end{equation}
                \begin{equation}
                VFP[{\rm _U}] =  VFM[{\rm _U}]
                \end{equation}
                \begin{equation}
                (QP[_E] = QM[_E])
                \end{equation}
                {\color{red}This option {\bf blows up} with big error at the boundaries.}
	\item[b]
                \begin{equation}
               (QP[_\rho]= QM[_\rho])
                \end{equation}
                \begin{equation}
                VFP[{\rm _U}] =  0
                \end{equation}
                \begin{equation}
                (QP[_E] = QM[_E])
                \end{equation}
                {\color{blue}This option {\bf does NOT blow up, BUT thermal boundary layer forms}.}
	\end{itemize}    
	
	\item Imposing Neumann on the background sensible temperature \overline{T}:
	       \begin{equation}
               (QP[_\rho]= QM[_\rho])
                \end{equation}
                \begin{equation}
                VFP[{\rm _T_{ref}}] = as Giraldo\&Restelli\,2008
                \end{equation}
                \begin{equation}
                (QP[_E] = QM[_E])
                \end{equation}
                  
{\color{red}This option {\bf blows up} with big error at the boundaries.}
	
\end{itemize}

Notes: the parenthesis indicate that things do not change whether we Include or not those condiditions.

\newpage

%\bibliographystyle{plainnat}
%\bibliography{bibliography_completa}

\end{document}











