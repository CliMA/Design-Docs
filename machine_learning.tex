% !TEX root = main.tex

\section{Machine Learning}
\label{sec:machine_learning}

In this section, we shall describe the role of the non-hydrostatic atmospheric models in the machine learning/data assimilation approach.  To make the exposition clear, let us represent the model in the following way
\[
\diff{\vc{q}}{t} = S_C(\vc{q}) + S_{NC}(\vc{q})
\]
where $\vc{q}$ is the solution vector, $S_C$ denotes the computable terms on the RHS of the equations while $S_{NC}$ denotes the non-computable parameters.  Examples of computable terms includes the full dry dynamics in addition to explicit convection.  Non-computable terms are those that are below the sub-grid scales of even an LES model.  These terms will be obtained from observational data.  Let us describe the nature of this data and how often it needs to be read in.

\subsection{Non-computable Terms}
What sort of data will we be reading in and how often? Once per simulation?  

\subsection{LES Model}
The LES model will require approximately 1 million DOF.  We are targeting 10-100 meter resolutions at this range.

\subsection{Global Model}
Need to describe the process of how to spin-off the LES models and use them to improve the global model. 200 km global resolutions or better if we can afford it. 